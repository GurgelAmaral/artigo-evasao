%
% Template for RBIE papers in LaTeX
%

% The above language combination is for this template document only.
% You should use one of the following:
\documentclass[english, spanish, brazilian]{RBIEarticle} % for papers in portuguese
%\documentclass[brazilian, spanish, english]{RBIEarticle} % for papers in english
%\documentclass[brazilian, english, spanish]{RBIEarticle} % for papers in spanish

% Papers in Portuguese or Spanish may require the following lines:
\usepackage[utf8]{inputenc} % chooses UTF-8 as the main character set
\usepackage[T1]{fontenc} % for correct syllable separation in accented words
% Pacotes para citações/referências ABNT
%usepackage[alf]{abntex2cite} % citações autor-data
%\usepackage[num]{abntex2cite} % citações numéricas
\usepackage{amsmath}

% The next two statements are needed for the example table in this document
% (i.e. you don't necessarily need them in your own paper)
\usepackage{colortbl}
\definecolor{gray}{gray}{.8}

% Citations and references (Biblatex)
% Citations and references (Biblatex)
\usepackage[style=abnt]{biblatex}
\usepackage{csquotes}
\addbibresource{references.bib}

% Here goes the paper main title
\title{Aprendizado de Máquina Aplicado à Predição da Evasão no Ensino Médio Brasileiro: Uma Abordagem Baseada em Variáveis Socioeconômicas}

% If the manuscript is written in English, then this element must be removed.
\titleinenglish{Machine Learning Applied to Predicting Dropout Rates in Brazilian High Schools: An Approach Based on Socioeconomic Variables}

% If the manuscript is written in English, then this element must be removed.
\titleinspanish{Aprendizaje automático aplicado a la predicción de la deserción escolar en la educación secundaria brasileña: un enfoque basado en variables socioeconómicas}

% Here goes the paper author information (repeat for two or more authors)
\author{%
	\parbox{3.8cm}{%
		Bruno Alexandre Dias da Silva\\
		Universidade de São Paulo\\
		ORCID: \href{https://orcid.org/0000-0000-0000-0000}{0000-0000-0000-0000}\\
		Brunoalexdias20@usp.br
	}
        \hspace{0.3cm}
	\parbox{3.8cm}{%
		Lucas Gurgel do Amaral\\
		Universidade de São Paulo\\
		ORCID: \href{https://orcid.org/0000-0000-0000-0000}{0000-0000-0000-0000}\\
		lucasgurgel@usp.br
	}
        \hspace{0.3cm}
        \parbox{3.8cm}{%
		Rafael de França\\
		Universidade de São Paulo\\
		ORCID: \href{https://orcid.org/0000-0000-0000-0000}{0000-0000-0000-0000}\\
		rafaeldefranca@usp.br
	}
        \hspace{0.3cm}
	\parbox{3.9cm}{\raggedright%
		Richard Pereira do Nascimento\\
		Universidade de São Paulo\\
		ORCID: \href{https://orcid.org/0000-0000-0000-0000}{0000-0000-0000-0000}\\
		rcdwoods@usp.br
	}
}

\Submission{dd/Mmm/yyyy}
\First_round_notif{dd/Mmm/yyyy}
\New_version{dd/Mmm/yyyy}
\Second_round_notif{dd/Mmm/yyyy}
\Camera_ready{dd/Mmm/yyyy}
\Edition_review{dd/Mmm/yyyy}
\Available_online{dd/Mmm/yyyy}
\Published{dd/Mmm/yyyy}

% Here goes the page heading information
\heading{Gurgel et al.
}{RBIE v.VV – 2025}

% And finally here goes the citation information
\citeas{Last name, Initials., \ldots \& Last name, Initials.  (Year). Article title in the original language. Revista Brasileira de Informática na Educação, vol, pp-pp. https://doi.org/10.5753/rbie.yyyy.id}

\citeas{
SILVA, B. A. D.; GURGEL, L.; FRANÇA, R. D.; NASCIMENTO, R. P. do. Aprendizado de Máquina Aplicado à Predição de Evasão no Ensino Médio em São Paulo. Revista Brasileira de Informática na Educação, vol , pp-pp, 2025.
}

%====================================================================
%\hyphenpenalty=10000
%\setcounter{page}{01}

\begin{document}
\maketitle

% If the manuscript is written in English, then this element must be removed.
\begin{otherlanguage}{brazilian}
\begin{abstract}
A evasão escolar no ensino brasileiro vem se mostrando como um grande desafio na qualificação e formação educacional do público infanto-juvenil, principalmente nas camadas menos favorecidas da sociedade brasileira. Jovens que não concluem o ensino médio não conseguem especializar-se em cursos superiores e, portanto, submetem-se a empregos com mão de obra barata, perdurando um ciclo de pobreza e baixos indicadores socioeconômicos. Para isso, este trabalho tem como objetivo avaliar a relevância das variáveis exclusivamente socioeconômicas e desenvolver modelos probabilísticos de aprendizado de máquina com o intuito de prever a probabilidade de um aluno evadir o ensino médio da rede de ensino de São Paulo baseado em técnicas como Floresta Aleatória, Árvores de Decisão, Regressão e Redes Neurais. Por meio da base de dados da Pesquisa Nacional por Amostra de Domicílios (PNAD) provenientes do Instituto Brasileiro de Geografia e Estatística (IBGE), serão extraídas variáveis que, de acordo com a literatura de estudos acerca de evasão, mais expliquem o abandono escolar contínuo. Os resultados deste artigo podem esclarecer o peso de fatores socioeconômicos enquanto fatores determinantes para a evasão, bem como auxiliar docentes e escolas a rastrearem e prestarem mais apoio àqueles alunos que apresentam alta chance de evasão por motivos socioeconômicos.  
\keywords\ Evasão escolar; Aprendizado de Máquina; PNAD; Variáveis Socioeconômicas; Ensino médio.
\end{abstract}
\end{otherlanguage}

\begin{otherlanguage}{english}
\begin{abstract}
School dropout rates in Brazil have proven to be a major challenge in the educational qualification and training of children and young people, especially in the less privileged segments of Brazilian society. Young people who do not complete high school are unable to specialize in higher education courses and, therefore, take on low-wage jobs, perpetuating a cycle of poverty and low socioeconomic indicators. Therefore, this study aims to assess the relevance of exclusively socioeconomic variables and develop probabilistic machine learning models to predict the probability of a student dropping out of high school in the São Paulo school system based on techniques such as Random Forest, Decision Trees, Regression, and Neural Networks. Using the database of the National Household Sample Survey (PNAD) from the Brazilian Institute of Geography and Statistics (IBGE), variables will be extracted that, according to the literature on dropout, best explain continuous school dropout. The results of this article may clarify the weight of socioeconomic factors as determinants of dropout, as well as help teachers and schools track and provide more support to those students who are at high risk of dropping out for socioeconomic reasons.  
\keywords\ School dropout; Machine learning; PNAD; Socioeconomic variables; Secondary education.
\end{abstract}
\end{otherlanguage}

% If the manuscript is written in English, then this element must be removed.
\begin{otherlanguage}{spanish}
\begin{abstract}
La deserción escolar en la educación brasileña se ha convertido en un gran desafío para la calificación y la formación educativa de los niños y jóvenes, especialmente en los sectores menos favorecidos de la sociedad brasileña. Los jóvenes que no completan la educación secundaria no pueden especializarse en cursos superiores y, por lo tanto, se ven obligados a aceptar empleos con mano de obra barata, perpetuando un ciclo de pobreza y bajos indicadores socioeconómicos. Por ello, este trabajo tiene como objetivo evaluar la relevancia de las variables exclusivamente socioeconómicas y desarrollar modelos probabilísticos de aprendizaje automático con el fin de predecir la probabilidad de que un estudiante abandone la educación secundaria en la red educativa de São Paulo, basándose en técnicas como bosques aleatorios, árboles de decisión, regresión y redes neuronales. A partir de la base de datos de la Encuesta Nacional por Muestra de Hogares (PNAD) del Instituto Brasileño de Geografía y Estadística (IBGE), se extraerán las variables que, según la literatura de estudios sobre el abandono escolar, mejor explican el abandono escolar continuo. Los resultados de este artículo pueden aclarar el peso de los factores socioeconómicos como factores determinantes de la deserción, así como ayudar a los docentes y a las escuelas a rastrear y brindar más apoyo a aquellos alumnos que presentan un alto riesgo de deserción por motivos socioeconómicos.  
\keywords\ Absentismo escolar; Aprendizaje automático; PNAD; Variables socioeconómicas; Educación secundaria.
\end{abstract}
\end{otherlanguage}

\pagebreak

%====================================================================

\section{Introdução}
A evasão escolar é definida como a ausência de retorno ao sistema de ensino formal do aluno em idade escolar após abandono ou reprovação. Deste modo, enquanto o abandono faz referência à situação do aluno que deixa de frequentar as aulas durante o ano letivo, a evasão refere-se ao aluno que, por qualquer motivo, não regressou à rede de ensino com o reinicio do ano letivo, assim como descrito pelo manual guia do Sistema de Alerta Preventivo  de Evasão e Abandono Escolar (SAP). 

Elucidada sua definição, a evasão se mostra um grande desafio na esfera do ensino brasileiro, em especial ao ensino público. Segundo os indicadores de taxa de transição e fluxo provenientes do censo escolar, a nível nacional e na etapa do ensino médio, enquanto a educação privada tem taxa de evasão de 2\% a 3\% na série histórica, escolas públicas estaduais apresentam evasão de 7\% a 12\% do contingente total de alunos. 

Quando analisado a nível municipal em São Paulo com base na série histórica de 2014 a 2021, também no ensino médio, a diferença é levemente superior. A rede privada apresenta taxa de 1\% a 2,5\% de evasão, ao passo que a rede estadual exibe taxa de 6\% a 12\%. Com base no último registro mensurado, em 2021, 8,7\% de alunos do ensino estadual se evadiram em comparação com 2,6 por cento do ensino privado. Para mais, proporcionalmente há um maior número de matrículas no ensino público em relação ao ensino privado, 340 mil e 81 mil em 2021 respectivamente segundo dados do INEP, o que reforça a gravidade do retorno ao ensino público. 

No estado de São Paulo, maior estado em termos econômicos, tanto no ensino público quanto privado, há taxa de evasão de, aproximadamente, 6\% a 9,8\% na série histórica de 2014 a 2021. No ano de 2021, ano com 1,6 milhão de estudantes matriculados, mais de 100000 (cem mil) alunos cometeram evasão, apenas no período descrito. 

Acerca das causas que levam à evasão, Silva (2013) ressalta que de modo geral o nível socioeconômico das famílias é um dos principais fatores que impele estudantes a não retomar a vida acadêmica. Um baixo nível socioeconômico, que reflete também a pobreza, induz indivíduos a abandonar os estudos à procura de trabalho para incorporar a renda familiar mensal; da mesma forma destaca-se a dificuldade em conciliar estudo e trabalho, procedente do contexto econômico-social do aluno. Silva (2013) também aponta que não há um conjunto de técnicas e sistemas governamentais para rastrear e entender o abandono e evasão escolar nas redes brasileiras de ensino, o que poderia ser aspecto-chave para a desinformação e, por conseguinte, à continuidade do problema.

Este artigo investiga a problemática da evasão no ensino brasileiro a partir da Pesquisa Nacional por Amostra de Domicílios (PNAD) com a finalidade de analisar a significância dos fatores socioeconômicos para a previsão e desenvolver modelos preditivos baseados em aprendizado de máquina para avaliar a probabilidade de evasão escolar a fim de proporcionar um método probabilístico para previnir evasão e abandono escolar e compreender como o perfil do estudante impacta seu desenvolvimento acadêmico. 


\section{Fundamentos Teóricos}

A evasão escolar é um fenômeno amplamente estudado nas áreas da Educação e das Ciências Sociais, podendo ser associado a fatores de ordem social, econômica e cultural. Segundo Araque, Roldán e Salguero (2009), as principais razões que influenciam o abandono da escola estão relacionadas a variáveis socioeconômicas, familiares, institucionais e psicológicas. Nesse sentido, compreender esse fenômeno requer considerar tanto o contexto social do aluno quanto as condições oferecidas pelo sistema educacional. 

No campo da análise de dados, o termo Mineração de Dados Educacionais (\textit{Educational Data Mining — EDM}) refere-se ao uso de métodos computacionais para explorar grandes volumes de dados gerados em ambientes educacionais, buscando padrões relevantes (BAKER; ISOTANI; CARVALHO, 2011). O objetivo disso é apoiar a tomada de decisão em políticas públicas e institucionais, fornecendo evidências que permitam compreender e reduzir problemas como a própria evasão escolar.

Associado à EDM, o Aprendizado de Máquina (\textit{Machine Learning}) é um subcampo da Inteligência Artificial que possibilita a criação de modelos preditivos a partir de dados. Esses modelos são capazes de identificar relações complexas entre variáveis e gerar previsões com base em informações históricas (MITCHELL, 1997). No contexto da evasão escolar, técnicas de aprendizado supervisionado permitem estimar a probabilidade de um estudante abandonar ou concluir seus estudos, dadas suas características socioeconômicas, acadêmicas e pessoais (TEODORO; KAPPEL, 2020).

Entre as abordagens mais comuns em problemas de classificação binária, destacam-se a Regressão Logística, as Redes Neurais e os métodos de Floresta Aleatória de Classificação. A Regressão Logística (RADÜNZ, 1992) permite modelar a relação entre variáveis independentes e uma variável dependente categórica, atribuindo probabilidades ao evento de interesse. Já as Redes Neurais, inspiradas no funcionamento do cérebro humano, são capazes de identificar padrões não-lineares e de alta complexidade (GARDNER; DORLING, 1998). Por fim, o método de Florestas Aleatórias, proposto por Breiman (2001), combina múltiplas árvores de decisão para melhorar a acurácia preditiva e reduzir problemas de sobreajuste (\textit{overfitting}).

Dessa forma, os fundamentos apresentados evidenciam a relevância de associar teorias sobre evasão escolar com métodos de mineração de dados e aprendizado de máquina, permitindo analisar grandes quantidades de dados, como a PNAD, e construir modelos preditivos capazes de apoiar políticas públicas e institucionais na área da Educação.

\section{Trabalhos Relacionados}
Diversos estudos buscaram modelar de forma probabilística ou com técnicas de aprendizado de máquina o fenômeno da evasão escolar no Brasil. Como exemplo, destacam dois trabalhos acerca da evasão no âmbito acadêmico.

Em Mello et al. (2023) foram realizadas análises exploratórias acerca das características consideradas (variáveis independentes) como coeficiente de rendimento, tipo de escola de origem, percentual de frequência, nível de ensino, cor ou raça, modalidade de curso, sexo, renda familiar bruta, situação acadêmica, dentre outros aspectos, de alunos do Instituto Federal do Pernambuco (IFPE). Com a análise constatou-se que, como alguns exemplos, alguns cursos e períodos apresentam evasão maior que outros; o percentual de evasão do sexo masculino era maior em comparação ao feminino. Contudo, não houve análise acerca da renda per capita dos indivíduos, que é reiterada por autores que exploraram o tema de forma qualitativa, como Silva (2013) e Ferreira \& Oliveira (2020). Os autores utilizaram como técnica de aprendizado de máquina para a previsão de evasão XGBoost, Florestas Aleatórias e Árvores de decisão, para predizer a situação de matrícula do estudante conforme a base de dados disponível, e obtiveram, respectivamente, 82\%, 83\% e 80\% de acurácia. Esses resultados sugerem que tais modelos apresentam ótimo desempenho preditivo para estimar as probabilidades de evasão escolar, em concordância com o objetivo deste trabalho.

Por outro lado, o trabalho de Teodoro e Kappel (2020) também segue a mesma lógica do trabalho supracitado, com a ressalva de que tem ênfase em atributos diferentes e apresenta uma análise exploratória mais robusta acerca das características correlacionadas à evasão. Teodoro e Kappel optaram por desenvolver modelos preditivos baseados em Naive Bayes, KNN, Árvores de Decisão, Florestas Aleatórias de Classificação e Redes Neurais, com acurácias de 60\%, 75\%, 77\%, 79\% e 78\%, respectivamente, no geral. A apuração mostra que, em termos de acurácia, Florestas Aleatórias e Redes Neurais são as técnicas mais adequadas para estimação de evasão. Porém, assim como o estudo anterior, tem como principal objeto de estudo a evasão no ensino superior.  

Nota-se que os estudos revisados não abordam a evasão do ensino médio nas escolas públicas. Há escassez de análise e implementação de variáveis acerca dos fatores socioeconômicos como o trabalho, a renda e a fome. Nesse sentido, o presente artigo pretende examinar e preencher estas lacunas, ao analisar e implementar atributos variados e propor modelos que expliquem de modo probabilístico o tópico.

\section{Metodologia}
A Pesquisa Nacional por Amostra de Domicílios (PNAD), realizada pelo Instituto Brasileiro de Geografia e Estatística (IBGE), é um conjunto de informações detalhadas sobre o cenário socioeconômico da sociedade brasileira. A partir dela, é possível extrair dados e informações das características gerais da população. A metodologia deste artigo consiste no processamento desses dados e a utilização de técnicas de aprendizado de máquina que possibilitem a previsão da probabilidade da evasão do aluno dadas as suas características de cunho social e econômico. Para tal, foram utilizadas as linguagens R, Python e a biblioteca Scikit-learn.

A base de dados foi obtida diretamente a partir do website do IBGE na seção PNAD Contínua e lida preliminarmente em R para interpretação dos dados de largura fixa a largura variável com o auxílio da biblioteca PNADcIBGE. Foram filtradas apenas os resultados pertinentes, a partir do de ano de 2016 até 2024, à analise de ocorrência de pessoas que não frequentam mais a escola, frequentaram escola alguma vez, cursaram como grau mais elevado o ensino médio regular ou 2º grau, e com idade até 18 anos. Com base nesses dados é possível rotular ocorrência de evasão ou conclusão do ensino médio como variável binária exclusiva dependente baseada no conclusão do curso.

Após o procedimento de filtragem, foi feita uma seleção das características, ou colunas, mais relevantes para explicar o fenômeno da evasão escolar. Características como sexo, cor ou raça, se já trabalhou ou estagiou por pelo menos 1 hora em alguma atividade remunerada em dinheiro, remunerada em mercadorias e bens, não remunerada ou atividade ocasional ("bico"), número de componentes do domicílio (exclusive as pessoas cuja condição no domicílio era pensionista, empregado doméstico ou parente do empregado doméstico), se recebe bolsa família ou outro auxílio governamental, rendimento domiciliar per capita
(habitual de todos os trabalhos e efetivo de outras fontes), situação do domicílio, tipo de área (do domicílio), horas trabalhadas por semana em atividade principal, secundária e ocasional, se houve procura de emprego pelo entrevistado nos últimos 30 dias, por quanto tempo procura emprego e o tipo de abastecimento de água do domicílio, as quais são algumas das características que mais se correlacionam qualitativamente ao abandono e à evasão escolar no ensino médio (FERREIRA; OLIVEIRA, 2020).

Foi gerado um arquivo CSV com os dados já filtrados e características selecionadas, e posteriormente este foi lido em Python para limpeza de dados faltantes, balanceamento de categorias pela técnica de sobreamostragem, e verificação dos tipos de variáveis com o objetivo de utilizá-lo como base de treinamento aos modelos de aprendizado de máquina. Foi também criada a coluna "evasao" codificada de forma binária para classificar o aluno como evasão (1) e não-evasão (0).


\subsection{Pré-processamento dos Dados}
Para simplificação e integridade dos modelos de regressão (OKEWOLE, 2012), as variáveis categóricas foram transformadas em variáveis binárias exclusivas. Variáveis numéricas foram padronizadas de acordo com a média das amostras e seu desvio padrão com base na fórmula:

\vspace{0.5cm}
\begin{equation}
\large Z = \frac{x-\mu}{s}
\end{equation}
\vspace{0.5cm}

Em que \textit{Z} é a novo valor da variável, \textit{x} é o valor atual da variável, $\mu$ representa a média da amostra para o grupo da variável (ou coluna) e \textit{s} é equivalente ao desvio-padrão do grupo da variável.

Para o treino e teste efetivo dos modelos, 85\% da base de dados foi destinada ao treinamento e o restante, 15\%, a testes para verificação das métricas. 

\subsection{Técnicas de Aprendizado de Máquina}
Este Trabalho apresenta três técnicas de aprendizado de máquina para efeito de modelagem probabilística, sendo elas Regressão Logística, Redes Neurais e Florestas Aleatórias de Classificação. Tais técnicas foram utilizadas devido a seus retornos baseados em estimativas em valores reais entre 0 e 1, que são consistentes com o modelo de probabilidades. Destaca-se que a escolha de diferentes técnicas reflete unicamente o objetivo de obter o mais acurado desempenho preditivo para o problema de previsão de probabilidades.

\subsection{Métricas de Desempenho}
Com o objetivo de extrair resultados acerca do desempenho dos modelos deste trabalho, foram utilizadas as fórmulas de precisão, revocação, acurácia e o índice F1 calculados com base na matriz de confusão proveniente do conjunto de testes com os modelos já treinados e selecionados os melhores via busca em grade de hiperparâmetros com validação cruzada em k-partições.


\subsubsection{Matriz de Confusão}
A matriz de confusão, assim como ressalta Franceschi (2019), é uma maneira de apresentar os valores reais e preditos pelo modelo com o intuito de mensurar os erros e acertos em cada classe de um determinado problema para se ter percepção dos seus desempenhos individuais.

Para este trabalho, dado que o problema segue a forma de classes binárias (evasão ou não-evasão), a matriz de confusão, seguindo o modelo de Franceschi (2019) é dada por:

\vspace{0.5cm}
\begin{equation}
\large
M = 
\begin{pmatrix}
VP & FP \\
FN & VN
\end{pmatrix}
\end{equation}
\vspace{0.5cm}

Conforme Castro e Braga (2011, apud Franceschi, 2019), a matriz principal da matriz representa os valores corretos preditos pelo modelo, isto é, \textit{VP} (Verdadeiro Positivo) e \textit{VN} (Verdadeiro Negativo). Os demais elementos expressam os erros retornados pelo modelo: \textit{FP} (Falso Positivo) e \textit{FN} (Falso Negativo).

Na representação deste trabalho, as linhas expressam as classes reais do problema, enquanto as colunas representam as classes preditas.

\subsubsection{Métricas Derivadas da Matriz de Confusão}
Dada a interpretação da matriz de confusão, pode-se, por meio de seus elementos, derivar métricas pertinentes à explicação do desempenho das classes e, por consequência, dos modelos deste trabalho, como demonstrado por Klén et al. (2024):

\vspace{0.5cm}
A taxa de precisão:
\begin{equation}
    \large \frac{VP}{VP+FP}
\end{equation}
\vspace{0.5cm}

\vspace{0.5cm}
A taxa de revocação:
\begin{equation}
    \large \frac{VP}{VP+FN}
\end{equation}
\vspace{0.5cm}

\vspace{0.5cm}
A taxa de acurácia:
\begin{equation}
    \large \frac{VP+VN}{VP+FP+VN+FN}
\end{equation}
\vspace{0.5cm}

\vspace{0.5cm}
E o índide F1:
\begin{equation}
    \large 2 \frac{P \cdot R}{P+R}
\end{equation}
\vspace{0.5cm}

Em que \textit{P} é a taxa de \textit{precisão} e \textit{R} a taxa de \textit{revocação}.

Acerca das métricas derivadas: a precisão representa os resultados classificados como verdadeiros positivos dentre os falsos e verdadeiros positivos, e mensura o quanto o algoritmo se distancia de falsos positivos; a taxa de revocação expressa a taxa de verdadeiros positivos em relação a verdadeiros positivos e falsos negativos, e demonstra o quanto o algoritmo separa corretamente verdadeiros positivos de falsos negativos; a acurácia representa a taxa de acertos bruta de todas as classes em todo o conjunto de teste; o índice F1, como descrito por Klén et al. (2024), é a média harmônica entre a precisão e a revocação. Para classes "negativas" (verdadeiro negativo), as mesmas lógicas da precisão e da revocação são implementadas com a ressalva de que são feitas em relação a falsos negativos e a falsos positivos respectivamente. 

\section{Resultados Parciais}

Com base nos dados coletados a partir da PNAD, as variáveis socioeconômicas, selecionadas neste presente trabalho, parecem não ser suficientes para determinar a evasão do aluno. O que se contrapõe aos trabalhos de Silva (2013) e Ferreira \& Oliveira (2020), que ressaltam tais variáveis como um dos principais motivos que levam ao abandono contínuo do ensino médio, em especial aos condições de trabalho de jovens em idade escolar.

Tal fato levanta a hipótese de que fatores socioeconômicos devem estar atrelados a fatores acadêmicos, como a presença ou notas, e fatores psicológicos, como a falta de interesse na escola, para uma real percepção e modelagem preditiva da evasão escolar. Por outro lado, também há a hipótese de que os fatores socioeconômicos que se correlacionam com a evasão não foram levados em consideração para a filtragem e modelagem deste trabalho.

A Pesquisa Nacional por Amostra de Domicílios (PNAD) apresenta um vasta base de dados acerca de aspectos socioeconômicos da população brasileira, contudo carece de determinantes acadêmicos específicos e questões que revelem a consideração do entrevistado acerca de sua vida no contexto acadêmico. 

Ao longo deste trabalho, as seguintes variáveis foram filtradas e tratadas para modelagem da evasão e treinamento posterior com os algoritmos de aprendizado de máquina:


\begin{table}
\caption{Descrição de Variáveis da Base de Dados}
\centering
\label{tab:variables}
\begin{tabular}{lp{8cm}}
\hline
Variável & Descrição \\ \hline
UF & Unidade da Federação do estudante \\
sexo & Sexo do estudante \\
raça & Raça/cor do estudante \\
idade & Idade em anos \\
frequenta\_escola & Indica se atualmente frequenta escola \\
frequentou\_escola & Indica se já frequentou escola em algum momento \\
curso\_frequentado & Tipo de curso frequentado (fundamental, médio, etc.) \\
terminou\_curso & Indica se concluiu o curso frequentado \\
V4001--V4004 & Variáveis da PNAD relacionadas ao trabalho e ocupação \\
condicao\_domicilio & Condição de ocupação do domicílio (próprio, alugado, cedido, etc.) \\
num\_pessoas\_domicilio & Número de pessoas no domicílio \\
bolsa\_familia & Recebimento de Bolsa Família \\
aux\_governo & Recebimento de outros auxílios governamentais \\
renda\_per\_capita & Renda domiciliar per capita \\
situacao\_domicilio & Situação do domicílio (urbano/rural) \\
tipo\_area\_dom & Tipo de área do domicílio \\
tipo\_domicilio & Tipo de construção do domicílio \\
tipo\_abastecimento\_de\_agua & Forma de abastecimento de água do domicílio \\
horas\_trabalhadas\_emp\_princ\_sem & Horas semanais trabalhadas no emprego principal \\
horas\_trabalhadas\_emp\_sec\_sem & Horas semanais trabalhadas no emprego secundário \\
horas\_trabalhadas\_outros\_emp\_sem & Horas semanais trabalhadas em outros empregos \\
procurou\_emprego\_ult\_30\_dias & Indica se procurou emprego nos últimos 30 dias \\
deseja\_trabalhar & Indica se deseja trabalhar atualmente \\
tempo\_procura\_emprego\_at1a & Tempo de procura por emprego: até 1 ano \\
tempo\_procura\_emprego\_1a2a & Tempo de procura por emprego: entre 1 e 2 anos \\
tempo\_procura\_emprego\_m2a & Tempo de procura por emprego: mais de 2 anos \\
num\_serie & Série escolar em que o estudante está matriculado \\
evasao & Indicador de evasão escolar (variável-alvo) \\ \hline
\end{tabular}
\end{table}


\subsubsection{Resultado do Desempenho dos Modelos}
A hipótese de que variáveis socioeconômicas não determinam de forma exclusiva o fenômeno da evasão no ensino médio pode ser constatada a partir do desempenho de cada classe com base nas métricas derivadas:

\vspace{0.5cm}
\begin{table}[htbp]
\centering
\label{tab:metricas}
\begin{tabular}{lcccc}
\hline
Classe        & Precisão & Revocação & Índice-F1 & Suporte \\ \hline
Não-evasão           & 0,86     & 0,77      & 0,81     & 2802    \\
Evasão           & 0,35     & 0,49      & 0,41     & 708     \\ \hline
Acurácia      &          &           & 0,72     & 3510    \\
Média Macro   & 0,60     & 0,63      & 0,61     & 3510    \\
Média Ponderada & 0,75   & 0,72      & 0,73     & 3510    \\ \hline
\end{tabular}
\caption{Métricas de desempenho do classificador Redes Neurais}
\end{table}
\vspace{0.5cm}

Com base na tabela, é possível perceber um alto desempenho em relação a Não-evasão, com uma precisão de 0,86, revocação de 0,77 e índice-f1 de 0,81. Contudo a classe que representa efetivamente a evasão tem precisão de 0,35 apenas, isto é, em todos os casos da classe Evasão (verdadeiros ou falsos), o algoritmo de Redes Neurais teve êxito em predizer corretamente apenas 35\% deles, e com falsos positivos (revocação) teve êxito na previsão de apenas 49\% dos casos. 

O algoritmo de Regressão Logística, apresentou desempenho semelhante para as classes:

\vspace{0.5cm}
\begin{table}[htbp]
\centering
\label{tab:metricas}
\begin{tabular}{lcccc}
\hline
Classe        & Precisão & Revocação & F1-Score & Suporte \\ \hline
Não-evasão           & 0,87     & 0,73      & 0,79     & 2802    \\
Evasão           & 0,35     & 0,57      & 0,43     & 708     \\ \hline
Acurácia      &          &           & 0,70     & 3510    \\
Média Macro   & 0,61     & 0,65      & 0,61     & 3510    \\
Média Ponderada & 0,76   & 0,70      & 0,72     & 3510    \\ \hline
\end{tabular}
\caption{Métricas de desempenho do classificador Regressão Logística}
\end{table}
\vspace{0.5cm}

O algoritmo de Floresta Aleatória se mostrou semelhante em relação aos outros modelos, com a ressalva de que apresentou desempenho ligeiramente pior em ambas as classes.

\vspace{0.5cm}
\begin{table}[htbp]
\centering
\label{tab:metricas2}
\begin{tabular}{lcccc}
\hline
Classe        & Precisão & Revocação & F1-Score & Suporte \\ \hline
Não-evasão           & 0,84     & 0,65      & 0,73     & 2802    \\
Evasão           & 0,27     & 0,53      & 0,36     & 708     \\ \hline
Acurácia      &          &           & 0,62     & 3510    \\
Média Macro   & 0,56     & 0,59      & 0,55     & 3510    \\
Média Ponderada & 0,73   & 0,62      & 0,66     & 3510    \\ \hline
\end{tabular}
\caption{Métricas de desempenho do classificador Floresta Aleatória}
\end{table}
\vspace{0.5cm}

Baseado nas tabelas de métricas dos diferentes modelos, apesar de evidenciado um baixo desempenho na classe de evasão, o algoritmo de Redes Neurais apresentou a maior acurácia global, contudo o modelo de Floresta Aleatória se mostrou o melhor na predição da classe evasão, com uma revocação e um índice-F1 superiores aos provenientes do modelo de Redes Neurais.

A natureza desbalanceada da base de dados bruta também pode ser um fator que interfere na capacidade do modelo de generalizar para ambas as classes, uma vez que para a classe de Evasão foi utilizado o método de sobreamostragem com a criação de registros sintéticos a partir de semelhança com k-vizinhos ($k=2$ neste trabalho)

\vspace{0.5cm}
\begin{table}[htbp]
\centering
\label{tab:proporcao}
\begin{tabular}{lc}
\hline
Classe & Proporção \\ \hline
Não-evasão  & 0,798 \\ 
Evasão      & 0,202 \\ \hline
\end{tabular}
\caption{Proporção das classes na base de dados}
\end{table}
\vspace{0.5cm}

Deste modo, pelo fato de aprender a partir de dados sintéticos, os diferentes modelos podem ter dificuldade de generalizar com dados reais, o que é verificado nos testes.


\pagebreak
\section{Cronograma}
O cronograma do projeto foi organizado de forma a contemplar as fases de pesquisa e os marcos de entrega definidos pela disciplina.
\begin{table}[h]
\centering
\begin{tabular}{p{3cm} p{7cm} p{5cm}}
\hline
\textbf{Período} & \textbf{Atividades principais} & \textbf{Marcos oficiais} \\
\hline
Agosto (até 20) & Revisão bibliográfica, definição do tema, objetivos e trabalhos relacionados & Entrega do plano de pesquisa (20/08) \\
Setembro        & Coleta de dados e pré-processamento & -- \\
Outubro (até 08) & Implementação inicial e testes dos modelos & Relatório parcial + vídeo (06--08/10) \\
Outubro (após 08) & Ajustes dos modelos, análises comparativas & -- \\
Novembro (até 19) & Consolidação dos resultados e redação final & Relatório final + apresentação parcial (17--19/11) \\
Novembro (após 19) & Revisão final e preparação da apresentação & Apresentação oficial (24--26/11) \\
\hline
\end{tabular}
\caption{Cronograma híbrido das atividades e entregas do projeto.}
\label{tab:cronograma}
\end{table}


%====================================================================


%See the guidelines for metadata and references:
%https://sol.sbc.org.br/journals/index.php/rbie/libraryFiles/downloadPublic/71
%====================================================================

\pagebreak
\nocite{*}
\printbibliography


\end{document}
