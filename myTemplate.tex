%
% Template for RBIE papers in LaTeX
%

% The above language combination is for this template document only.
% You should use one of the following:
\documentclass[english, spanish, brazilian]{RBIEarticle} % for papers in portuguese
%\documentclass[brazilian, spanish, english]{RBIEarticle} % for papers in english
%\documentclass[brazilian, english, spanish]{RBIEarticle} % for papers in spanish

% Papers in Portuguese or Spanish may require the following lines:
\usepackage[utf8]{inputenc} % chooses UTF-8 as the main character set
\usepackage[T1]{fontenc} % for correct syllable separation in accented words
% Pacotes para citações/referências ABNT
%usepackage[alf]{abntex2cite} % citações autor-data
%\usepackage[num]{abntex2cite} % citações numéricas
\usepackage{amsmath}

% The next two statements are needed for the example table in this document
% (i.e. you don't necessarily need them in your own paper)
\usepackage{colortbl}
\definecolor{gray}{gray}{.8}

% Citations and references (Biblatex)
% Citations and references (Biblatex)
\usepackage[style=abnt]{biblatex}
\usepackage{csquotes}
\addbibresource{references.bib}

% Here goes the paper main title
\title{Modelagem Estatística da Evasão no Ensino Médio Brasileiro: Uma Investigação Baseada em Variáveis Institucionais e Socioeconômicas}

% If the manuscript is written in English, then this element must be removed.
\titleinenglish{Statistical Modeling of Dropout Rates in Brazilian High Schools: An Investigation Based on Institutional and Socioeconomic Variables}

% If the manuscript is written in English, then this element must be removed.
\titleinspanish{Modelización estadística de la deserción escolar en la enseñanza secundaria brasileña: una investigación basada en variables institucionales y socioeconómicas}

% Here goes the paper author information (repeat for two or more authors)
\author{%
	\parbox{3.8cm}{%
		Bruno Alexandre Dias da Silva\\
		Universidade de São Paulo\\
		ORCID: \href{https://orcid.org/0000-0000-0000-0000}{0000-0000-0000-0000}\\
		brunoalexdias20@usp.br
	}
        \hspace{0.3cm}
	\parbox{3.8cm}{%
		Lucas Gurgel do Amaral\\
		Universidade de São Paulo\\
		ORCID: \href{https://orcid.org/0000-0000-0000-0000}{0000-0000-0000-0000}\\
		lucasgurgel@usp.br
	}
        \hspace{0.3cm}
        \parbox{3.8cm}{%
		Rafael de França\\
		Universidade de São Paulo\\
		ORCID: \href{https://orcid.org/0000-0000-0000-0000}{0000-0000-0000-0000}\\
		rafaeldefranca@usp.br
	}
        \hspace{0.3cm}
	\parbox{3.9cm}{\raggedright%
		Richard Pereira do Nascimento\\
		Universidade de São Paulo\\
		ORCID: \href{https://orcid.org/0000-0000-0000-0000}{0000-0000-0000-0000}\\
		rcdwoods@usp.br
	}
}

\Submission{dd/Mmm/yyyy}
\First_round_notif{dd/Mmm/yyyy}
\New_version{dd/Mmm/yyyy}
\Second_round_notif{dd/Mmm/yyyy}
\Camera_ready{dd/Mmm/yyyy}
\Edition_review{dd/Mmm/yyyy}
\Available_online{dd/Mmm/yyyy}
\Published{dd/Mmm/yyyy}

% Here goes the page heading information
\heading{Gurgel et al.
}{RBIE v.VV – 2025}

% And finally here goes the citation information
\citeas{Last name, Initials., \ldots \& Last name, Initials.  (Year). Article title in the original language. Revista Brasileira de Informática na Educação, vol, pp-pp. https://doi.org/10.5753/rbie.yyyy.id}

\citeas{
SILVA, B. A. D.; GURGEL, L.; FRANÇA, R. D.; NASCIMENTO, R. P. do. Aprendizado de Máquina Aplicado à Predição de Evasão no Ensino Médio em São Paulo. Revista Brasileira de Informática na Educação, vol , pp-pp, 2025.
}

%====================================================================
%\hyphenpenalty=10000
%\setcounter{page}{01}

\begin{document}
\maketitle

% If the manuscript is written in English, then this element must be removed.
\begin{otherlanguage}{brazilian}
\begin{abstract}
A evasão escolar no ensino brasileiro vem se mostrando como um grande desafio na formação educacional de crianças e jovens, principalmente nas camadas menos favorecidas da sociedade brasileira. Jovens que não concluem o ensino médio não conseguem se especializar em cursos superiores e, portanto, ocupam empregos de baixa remuneração, perpetuando um ciclo de pobreza e baixos indicadores socioeconômicos. Para isso, este trabalho tem como objetivo avaliar e comparar a relevância de variáveis socioeconômicas e institucionais, a fim de desenvolver modelos probabilísticos de aprendizado de máquina capazes de estimar a probabilidade de evasão escolar no ensino médio público brasileiro. A análise é realizada com base em Regressão Linear Múltipla e avaliada por métricas como R² e erro quadrático médio. Por meio da base de dados de Indicadores Educacionais e microdados do Censo Escolar do Instituto Nacional de Estudos e Pesquisas Educacionais Anísio Teixeira (INEP) e da base de dados da Pesquisa Nacional por Amostra de Domicílios (PNAD), provenientes do Instituto Brasileiro de Geografia e Estatística (IBGE), serão extraídas variáveis que, de acordo com a literatura sobre evasão, mais expliquem o abandono escolar contínuo. Os resultados deste artigo podem auxiliar docentes e escolas a identificarem e fornecerem mais apoio aos alunos que apresentam alta chance de evasão por motivos de cunho social, econômico, acadêmico e/ou institucional.
\keywords\ Evasão escolar; Aprendizado de máquina; PNAD; Variáveis Socioeconômicas; Variáveis Institucionais; Inep; Ensino médio.
\end{abstract}
\end{otherlanguage}

\begin{otherlanguage}{english}
\begin{abstract}
School dropout rates in Brazil have proven to be a major challenge in the educational qualification and training of children and young people, especially in the less privileged segments of Brazilian society. Young people who do not complete high school are unable to specialize in higher education courses and, therefore, take on low-wage jobs, perpetuating a cycle of poverty and low socioeconomic indicators. Therefore, this study aims to evaluate and compare the significance of socioeconomic and institutional variables in order to develop probabilistic machine learning models capable of estimating the probability of a student dropping out of high school in the Brazilian school system. The models are evaluated using appropriate statistical metrics, such as the coefficient of determination (R²) and the mean squared error (MSE), taking Multiple Linear Regression as a reference approach. Using data from the Educational Indicators and microdata from the School Census of the Anísio Teixeira National Institute for Educational Studies and Research (INEP), and from the Brazilian Institute of Geography and Statistics (IBGE) National Household Sample Survey (PNAD), variables will be extracted that, according to the literature on dropout, best explain continuous school abandonment. The results of this article can help teachers and schools track and provide more support to students at high risk of dropping out due to social, economic, academic, and/or institutional reasons.
\keywords\ School dropout; Machine learning; PNAD; Socioeconomic variables; Institutional variables; Inep ;Secondary education.
\end{abstract}
\end{otherlanguage}

% If the manuscript is written in English, then this element must be removed.
\begin{otherlanguage}{spanish}
\begin{abstract}
La deserción escolar en la educación brasileña se ha convertido en un gran desafío para la formación educativa de niños y jóvenes, especialmente en los sectores menos favorecidos de la sociedad brasileña. Los jóvenes que no completan la educación secundaria no pueden especializarse en cursos superiores y, por lo tanto, ocupan empleos de baja remuneración, perpetuando un ciclo de pobreza y bajos indicadores socioeconómicos. Por ello, este estudio tiene como objetivo evaluar y comparar variables socioeconómicas e institucionales para desarrollar modelos probabilísticos de aprendizaje automático capaces de estimar la probabilidad de deserción escolar en la educación secundaria pública brasileña. El análisis se basa en la Regresión Lineal Múltiple y se evalúa mediante métricas como R² y el error cuadrático medio. A partir de la base de microdatos del Censo Escolar y de los Indicadores Educativos del Instituto Nacional de Estudios e Investigaciones Educativas Anísio Teixeira (INEP) y de la base de datos de la Encuesta Nacional por Muestra de Hogares (PNAD) del Instituto Brasileño de Geografía y Estadística (IBGE), se extraerán variables que, según la literatura sobre deserción escolar, mejor explican el abandono escolar continuo. Los resultados de este artículo pueden ayudar a los docentes y a las escuelas a identificar y brindar más apoyo a los estudiantes con alto riesgo de deserción por motivos de índole social, económica, académica y/o institucional.
\keywords\ Absentismo escolar; Aprendizaje automático; PNAD; Variables socioeconómicas; Educación secundaria.
\end{abstract}
\end{otherlanguage}

\pagebreak

%====================================================================

\section{Introdução}

A evasão escolar constitui um dos principais desafios para a educação brasileira, representando não apenas a interrupção de trajetórias individuais, mas também a perpetuação de desigualdades sociais e o comprometimento do desenvolvimento econômico do país. No contexto do ensino médio, etapa final da educação básica, o fenômeno assume contornos particularmente preocupantes: segundo indicadores de fluxo escolar do Censo Escolar (INEP, 2023), o ensino médio apresenta taxa de abandono de aproximadamente 6\%, com disparidades significativas entre as redes de ensino. A rede pública concentra os maiores índices de evasão, enquanto a rede privada apresenta taxas substancialmente inferiores, evidenciando desigualdades estruturais que ultrapassam questões meramente pedagógicas.

Compreender os fatores associados à evasão exige reconhecer sua natureza multidimensional. A literatura nacional e internacional aponta para a convergência de elementos socioeconômicos — como baixa renda familiar, trabalho precoce e insegurança alimentar —, fatores acadêmicos — repetência, distorção idade-série e baixo desempenho — e variáveis institucionais — infraestrutura escolar, formação docente e tamanho de turmas (SILVA, 2016; ARAQUE; ROLDÁN; SALGUERO, 2009). Essa complexidade demanda abordagens analíticas capazes de quantificar a contribuição relativa de cada dimensão e orientar políticas públicas baseadas em evidências.

Dados do IBGE (2024) mostram que cerca de 8,7 milhões de jovens entre 14 e 29 anos abandonaram os estudos ou nunca frequentaram a escola, sendo que parcela significativa não havia concluído o ensino médio. As consequências desse abandono são duradouras: a renda média de trabalhadores com ensino médio completo é significativamente superior à de quem abandonou os estudos antes dessa etapa, perpetuando ciclos de vulnerabilidade e limitando oportunidades de mobilidade social. Em perspectiva comparada, o Brasil apresenta taxas de abandono escolar superiores à média de países da América Latina e muito distantes das registradas em nações com sistemas educacionais mais consolidados, evidenciando a necessidade urgente de políticas públicas de permanência e combate à evasão.

Apesar da relevância do tema, ainda persiste lacuna de estudos que integrem, em escala nacional, dados educacionais e socioeconômicos para análise sistemática dos determinantes da evasão. Grande parte da literatura brasileira concentra-se em recortes locais ou institucionais, com ênfase em modelos preditivos de risco individual voltados ao ensino superior. Investigações que articulem bases de dados oficiais — como os microdados do Censo Escolar, do Sistema de Avaliação da Educação Básica (SAEB) e da Pesquisa Nacional por Amostra de Domicílios (PNAD) — para analisar o fenômeno em nível municipal, identificando padrões territoriais e desigualdades regionais, permanecem escassas.

Nesse contexto, este trabalho propõe uma análise explicativa da evasão no ensino médio brasileiro a partir de modelagem estatística baseada em Regressão Linear Múltipla. O objetivo é investigar em que medida variáveis institucionais — como formação docente, infraestrutura e composição de turmas —, variáveis acadêmicas — taxa de repetência, distorção idade-série e desempenho em avaliações — e variáveis socioeconômicas — renda per capita, trabalho precoce e acesso a programas sociais — contribuem para explicar a variação das taxas de evasão entre municípios brasileiros.

A escolha da Regressão Linear Múltipla justifica-se por sua capacidade de quantificar o impacto individual de cada variável sobre a taxa de evasão, oferecendo coeficientes diretamente interpretáveis que facilitam a compreensão dos mecanismos subjacentes ao fenômeno e a aplicação dos resultados em políticas educacionais. Diferentemente de abordagens orientadas à predição de risco individual, este estudo privilegia a análise agregada em nível municipal, permitindo identificar contextos territoriais de maior vulnerabilidade e orientar estratégias de intervenção direcionadas.

Para tanto, foram integradas bases de dados do Instituto Nacional de Estudos e Pesquisas Educacionais Anísio Teixeira (INEP) — incluindo indicadores educacionais do Censo Escolar e microdados do SAEB — e da Pesquisa Nacional por Amostra de Domicílios (PNAD) do IBGE. Essa articulação possibilita examinar simultaneamente dimensões escolares e sociais, ampliando a compreensão sobre como fatores estruturais, pedagógicos e socioeconômicos interagem na determinação da evasão escolar no ensino médio brasileiro. Os resultados obtidos visam subsidiar gestores públicos, pesquisadores e educadores na formulação de políticas mais efetivas para a redução da evasão e promoção da equidade educacional.



\section{Fundamentos Teóricos}

O presente capítulo tem como objetivo apresentar a base conceitual que sustenta esta pesquisa, fornecendo o embasamento necessário para compreender as principais variáveis que influenciam a evasão escolar no Brasil em nível municipal. A fundamentação teórica busca situar o estudo no contexto das pesquisas já existentes, permitindo identificar os fatores socioeconômicos, educacionais e estruturais que impactam a permanência dos estudantes, bem como os modelos e abordagens utilizados para analisar tais fenômenos. Além disso, o capítulo descreve os fundamentos teóricos, as bases de dados e as técnicas estatísticas e computacionais empregadas na pesquisa, estabelecendo o alicerce metodológico para as análises realizadas nos capítulos seguintes.
\subsection{Evasão Escolar e Fatores Socioeconômicos}
A evasão escolar é um fenômeno amplamente estudado nas áreas da Educação e das Ciências Sociais, podendo ser associado a fatores de ordem social, econômica e cultural. Segundo Araque, Roldán e Salguero (2009), as principais razões que influenciam o abandono da escola estão relacionadas a variáveis socioeconômicas, familiares, institucionais e psicológicas. Nesse sentido, compreender esse fenômeno requer considerar tanto o contexto social do aluno quanto as condições oferecidas pelo sistema educacional.

\subsection{Mineração de Dados Educacionais e Aprendizado de Máquina}
No campo da análise de dados, o termo Mineração de Dados Educacionais (\textit{Educational Data Mining — EDM}) refere-se ao uso de métodos computacionais para explorar grandes volumes de dados gerados em ambientes educacionais, buscando padrões relevantes (BAKER; ISOTANI; CARVALHO, 2011). O objetivo disso é apoiar a tomada de decisão em políticas públicas e institucionais, fornecendo evidências que permitam compreender e reduzir problemas como a própria evasão escolar.

Associado à EDM, o Aprendizado de Máquina (\textit{Machine Learning}) é um subcampo da Inteligência Artificial que possibilita a criação de modelos preditivos a partir de dados. Esses modelos são capazes de identificar relações complexas entre variáveis e gerar previsões com base em informações históricas (MITCHELL, 1997). No contexto da evasão escolar, técnicas de aprendizado supervisionado permitem estimar a probabilidade de um estudante abandonar ou concluir seus estudos, dadas suas características socioeconômicas, acadêmicas e pessoais (TEODORO; KAPPEL, 2020).

\subsection{Principais Técnicas Utilizadas}
Entre as abordagens utilizadas para a análise de fatores que influenciam fenômenos complexos, destaca-se a Regressão Linear Múltipla, que permite investigar a relação entre uma variável dependente contínua e múltiplas variáveis independentes. Esse método possibilita compreender de forma quantitativa o impacto de diferentes fatores sobre o resultado observado, identificando quais variáveis apresentam maior influência e em que magnitude (MONTGOMERY; PECK; VINING, 2012).
No contexto da evasão escolar, a Regressão Linear Múltipla é aplicada para estimar como aspectos socioeconômicos, educacionais e estruturais contribuem para a variação das taxas de evasão entre municípios, permitindo uma análise explicativa e comparativa das desigualdades regionais.

\subsection{Técnicas Computacionais Complementares}
Embora este trabalho foque exclusivamente na aplicação da Regressão Linear Múltipla, outras técnicas computacionais também têm sido amplamente empregadas em estudos voltados à predição e análise de dados educacionais. Entre essas abordagens, destacam-se métodos como Regressão Logística, Redes Neurais Artificiais e Florestas Aleatórias de Classificação, amplamente utilizados em pesquisas que buscam estimar a probabilidade de evasão escolar e identificar padrões complexos de comportamento estudantil (HOSMER; LEMESHOW, 2000; GARDNER; DORLING, 1998; BREIMAN, 2001).

Apesar da diversidade de métodos existentes, a escolha da Regressão Linear Múltipla neste estudo se justifica por sua capacidade de quantificar o impacto individual de múltiplas variáveis explicativas sobre a taxa de evasão escolar, permitindo identificar quais fatores exercem maior influência sobre o fenômeno. Essa abordagem fornece não apenas um modelo preditivo, mas também uma interpretação direta dos coeficientes, facilitando a compreensão e a aplicação dos resultados em políticas públicas e estratégias educacionais.

\subsection{A Pesquisa Nacional por Amostra de Domicílios (PNAD)}
A Pesquisa Nacional por Amostra de Domicílios (PNAD), realizada pelo Instituto Brasileiro de Geografia e Estatística (IBGE), constitui uma das principais fontes de dados socioeconômicos no Brasil. Seu objetivo é coletar informações abrangentes sobre características demográficas, educacionais, ocupacionais e de rendimento da população brasileira, por meio de entrevistas domiciliares aplicadas em amostras representativas em nível nacional, regional e estadual (IBGE, 2022).

A PNAD Contínua, em vigor desde 2012, aprimorou o levantamento ao adotar coleta trimestral, permitindo análises mais atualizadas e consistentes acerca da dinâmica social e econômica do país. Entre suas variáveis, destacam-se renda familiar per capita, inserção no mercado de trabalho, características do domicílio, composição familiar, escolaridade e acesso a programas sociais. Tais informações são de grande relevância para estudos sobre evasão escolar, pois possibilitam identificar relações entre vulnerabilidade socioeconômica e permanência na escola.

\subsection{Instituto Nacional de Estudos e Pesquisas Educacionais Anísio Teixeira (INEP)}
O Instituto Nacional de Estudos e Pesquisas Educacionais Anísio Teixeira (INEP), autarquia federal vinculada ao Ministério da Educação (MEC), tem como missão promover estudos, pesquisas e avaliações sobre o sistema educacional brasileiro. Fundado em 1937, o INEP é responsável pela produção e disseminação de informações estatísticas e avaliativas que subsidiam a formulação e o monitoramento de políticas públicas educacionais em âmbito nacional (INEP, 2023).

O Instituto realiza levantamentos censitários anuais em todas as etapas da educação básica e superior, além de coordenar avaliações de larga escala, como o Sistema de Avaliação da Educação Básica (SAEB) e o Exame Nacional do Ensino Médio (ENEM). Os dados produzidos pelo INEP constituem a principal fonte oficial de informações sobre matrícula, rendimento escolar, infraestrutura, profissionais da educação e fluxo escolar no país, sendo essenciais para a análise de fenômenos como evasão, repetência e distorção idade-série.

\subsection{Microdados do Censo Escolar}
O Censo Escolar, realizado anualmente pelo INEP em regime de colaboração com as secretarias estaduais e municipais de educação, constitui o principal instrumento de coleta de informações sobre a educação básica brasileira. Seu objetivo é reunir dados detalhados sobre estudantes, turmas, escolas e profissionais da educação em todas as redes de ensino – públicas e privadas –, abrangendo as etapas de educação infantil, ensino fundamental e ensino médio (INEP, 2023).

Os microdados do Censo Escolar são disponibilizados publicamente pelo INEP e permitem análises granulares sobre características individuais dos estudantes, como idade, sexo, raça/cor, situação de matrícula, modalidade de ensino e deficiências, além de informações sobre infraestrutura escolar, localização, dependência administrativa e recursos disponíveis. Esses dados são fundamentais para estudos sobre evasão escolar, pois possibilitam acompanhar longitudinalmente a trajetória estudantil, identificar fatores associados ao abandono e à reprovação, bem como analisar disparidades regionais e socioeconômicas que impactam o acesso e a permanência na escola.

\subsection{Sistema de Avaliação da Educação Básica (SAEB)}
O Sistema de Avaliação da Educação Básica (SAEB), também coordenado pelo INEP, constitui uma das principais avaliações externas e padronizadas da qualidade da educação no Brasil. Instituído em 1990, o SAEB tem como objetivo diagnosticar a educação básica brasileira e produzir indicadores sobre o desempenho dos estudantes em Língua Portuguesa e Matemática, além de coletar informações contextuais sobre condições de aprendizagem, perfil docente e clima escolar (INEP, 2023).

A partir de 2019, o SAEB passou a ser censitário para estudantes dos anos finais do ensino fundamental e do ensino médio, permitindo avaliações em nível de escola, município e estado. Os microdados do SAEB, disponibilizados junto aos do Censo Escolar, incluem resultados de proficiência, questionários contextuais aplicados a estudantes, professores e diretores, além de indicadores socioeconômicos e de infraestrutura. Tais informações são relevantes para estudos sobre evasão escolar, pois possibilitam relacionar desempenho acadêmico, contexto familiar e escolar com a permanência ou abandono dos estudantes, oferecendo subsídios para políticas educacionais mais direcionadas e eficazes.

\subsection{Síntese}
Dessa forma, os fundamentos apresentados evidenciam a relevância de associar teorias sobre evasão escolar com métodos de mineração de dados e aprendizado de máquina, bem como o uso de bases de dados socioeconômicas e educacionais amplas como a PNAD, Censo Escolar e SAEB. Essa integração permite analisar grandes quantidades de dados e construir modelos preditivos capazes de apoiar políticas públicas e institucionais na área da Educação, contribuindo para a redução da evasão escolar e para a promoção da equidade educacional.

\section{Trabalhos Relacionados}
A evasão escolar é um fenômeno complexo e multifatorial que tem sido amplamente estudado em diferentes níveis e contextos educacionais. Pesquisas recentes convergem quanto à importância de combinar variáveis institucionais e socioeconômicas para compreender o abandono e a permanência escolar, especialmente no ensino médio. O consenso entre os autores é que as decisões de evasão não se explicam apenas por características individuais, mas refletem também a estrutura e as condições das escolas, bem como fatores externos relacionados à renda e às oportunidades sociais.

\textcite{lopesfilho2021deteccao} analisaram dados administrativos da rede pública paulista com o objetivo de identificar estudantes em risco de evasão. Embora o trabalho utilize métodos de aprendizado de máquina, ele destaca a relevância de fatores contextuais ao nível da escola, como o tamanho das turmas, a razão aluno-professor e o desempenho médio da instituição. Os resultados mostram que escolas com turmas mais numerosas e menor desempenho tendem a apresentar taxas mais elevadas de abandono. Tais achados embasam a escolha, neste estudo, de variáveis como a \textit{média de alunos por turma} e o \textit{esforço docente}, que refletem a capacidade estrutural e a sobrecarga dos professores no ambiente escolar.

De modo complementar, a revisão sistemática conduzida por \textcite{banaag2024factors} identifica determinantes internos à escola como alguns dos principais preditores de evasão, destacando a baixa qualificação docente, as práticas pedagógicas inadequadas, a defasagem idade-série e a reprovação recorrente. Essas evidências sustentam a inclusão, neste trabalho, de variáveis que expressam o desempenho e a qualificação docente, como a \textit{taxa de reprovação} e a \textit{proporção de professores com formação superior}. Ambas representam dimensões críticas do processo de ensino-aprendizagem e ajudam a compreender a influência da qualidade da docência sobre a permanência dos alunos.

Em âmbito nacional, \textcite{shirasu2018determinantes} aplicaram modelos econométricos multiníveis ao ensino médio do Ceará, observando que repetência e atraso escolar elevam significativamente a probabilidade de abandono, enquanto políticas de transferência de renda, como o Bolsa Família, reduzem esse risco. O estudo reforça que as condições socioeconômicas e institucionais atuam de forma interdependente e que o enfrentamento da evasão requer abordagens que contemplem ambas as dimensões. Essa conclusão fundamenta a integração, neste artigo, de indicadores provenientes tanto do INEP quanto da PNAD, permitindo relacionar variáveis estruturais das escolas com características econômicas e sociais dos estudantes e de seus domicílios.

A literatura também evidencia que a infraestrutura física da escola exerce papel relevante na permanência estudantil. \textcite{coc2024infraestrutura} destacam que ambientes precários, ausência de laboratórios e de recursos tecnológicos, falta de manutenção e limitações no acesso à Internet reduzem o engajamento e aumentam as taxas de evasão. Nesse sentido, a inclusão de variáveis de \textit{infraestrutura escolar} — como presença de laboratório de ciências e informática, biblioteca, quadra esportiva e conectividade — permite avaliar o impacto das condições materiais sobre a probabilidade de abandono. Esses elementos complementam as dimensões institucionais e docentes, ampliando a capacidade explicativa do modelo.

Ainda que o presente estudo se concentre no ensino médio, pesquisas de outros níveis educacionais reforçam a relevância de abordagens quantitativas voltadas à interpretação de efeitos marginais entre variáveis institucionais. \textcite{oliveira2024evasaoead}, ao analisarem a evasão em cursos superiores a distância, empregaram regressão linear múltipla para estimar o impacto de fatores como idade, gênero e tipo de escola de origem sobre as taxas de desistência. O uso da regressão permitiu mensurar a contribuição de cada variável e avaliar o desempenho do modelo por meio do \textit{erre ao quadrado} (R²), que indica o quanto da variação da evasão é explicada pelos regressores, e do \textit{erro quadrático médio} (MSE), que expressa a magnitude média dos erros de previsão. Essa abordagem reforça a adequação do método estatístico adotado neste trabalho, cujo foco recai sobre a interpretação e significância das variáveis independentes.

De modo geral, os estudos revisados demonstram que fatores estruturais e institucionais — como o tamanho das turmas, o esforço docente e a infraestrutura — combinam-se a aspectos acadêmicos e socioeconômicos na explicação do abandono escolar. A literatura revisada também mostra que análises de regressão são adequadas para quantificar e interpretar o peso relativo desses determinantes, oferecendo medidas objetivas do impacto de cada variável sobre a taxa de evasão. 

Dessa forma, as evidências apontam três diretrizes principais que orientam o presente trabalho: (i) \textit{variáveis institucionais e escolares são determinantes da evasão}, uma vez que condições estruturais e organizacionais das escolas influenciam o engajamento e a permanência dos alunos; (ii) \textit{indicadores de desempenho e qualificação docente explicam parte substancial do fenômeno}, pois refletem a qualidade do ensino e o acompanhamento pedagógico; e (iii) \textit{condições socioeconômicas e materiais interagem com o contexto escolar}, reforçando a importância de integrar dados do INEP e da PNAD em um mesmo modelo explicativo. 

Com base nessas evidências, o presente artigo propõe uma abordagem estatística voltada à mensuração e interpretação dos efeitos das variáveis institucionais e socioeconômicas sobre a evasão no ensino médio brasileiro. Essa fundamentação teórica e empírica orienta a construção da base de dados e sustenta as escolhas metodológicas descritas na próxima seção, que detalha os procedimentos de coleta, tratamento e modelagem utilizados na pesquisa.

\section{Metodologia}
O Instituto Nacional de Estudos e Pesquisas Educacionais Anísio Teixeira (INEP) dispõe uma vasta base de microdados e dados abertos baseados em variáveis institucionais, acadêmicas e socioeconômicas do sistema de ensino brasileiro no decorrer das últimas 2 décadas. Como complemento, a Pesquisa Nacional por Amostra de Domicílios (PNAD), realizada pelo Instituto Brasileiro de Geografia e Estatística (IBGE), é um conjunto de informações detalhadas sobre o cenário socioeconômico da sociedade brasileira. A partir dela, é possível extrair dados e informações das características gerais da população.  

Apoiado em tais conjuntos de dados, a metodologia deste artigo consiste no processamento desses dados e a utilização de técnicas estatísticas que possibilitem uma análise quantitativa acerca da evasão de alunos do ensino médio brasileiro, e quais fatores explicam tal fenômeno, dadas as suas características de cunho social, econômico, acadêmico e institucional. Para tal, foram utilizadas as linguagens R, Python e a biblioteca Statsmodels.  

Conjuntos de dados provenientes do INEP foram extraídos do site de dados abertos do instituto na aba de "Indicadores Educacionais" e "Microdados". Bases de dados como a taxa de repetência e de evasão, taxa de distorção idade-série, índice do esforço docente, número médio de alunos por turma, nível de adequação do docente e microdados da educação básica (todos a nível municipal e do ano de 2017) foram coletados, extraindo apenas ocorrências do ensino médio, em formato de planilhas xlsx ou arquivo com valores separados por vírgula (\textit{CSV}) e lidos posteriormente em Python com a biblioteca de manipulação de dados Pandas.

Em sequência, foi realizada a junção das bases supracitadas por meio das colunas, renomeadas a partir da amostra original, UF (sigla da unidade federativa), COD\_MUNICIPIO (código do município), LOCALIZACAO (urbana ou rural) E DEP\_ADMINISTRATIVA (instituição privada ou pública), consideradas chave primária composta do conjunto de dados. Deste modo, foi construído uma única estrutura de dados tabular (\textit{dataframe}) contendo as variáveis institucionais relevantes à manifestação da evasão na fase final da educação básica fundamento na literatura da evasão escolar no Brasil e no mundo (BANAAG ET AL., 2024; SHIRASU \& ARRAES, 2018; SOUSA ET AL., 2025). 

Para modelagem estatística ulterior isolada também foi coletada uma amostra a nível escolar do Sistema de Avaliação da Educação Básica (SAEB), com variáveis sociais e acadêmicas no ano de 2017, também oriunda dos microdados do INEP.

A base de dados da PNAD foi obtida diretamente a partir do website do IBGE na seção PNAD Contínua e lida preliminarmente em R para interpretação dos dados de largura fixa a largura variável com o auxílio da biblioteca PNADcIBGE. Foram filtrados apenas os resultados pertinentes, a partir do de ano de 2016 até 2024, à analise de ocorrência de pessoas que não frequentam mais a escola, frequentaram escola alguma vez, cursaram como grau mais elevado o ensino médio regular ou 2º grau, e com idade até 18 anos. Com base nesses dados é possível rotular ocorrência de evasão ou conclusão do ensino médio como variável binária exclusiva dependente baseada na conclusão do curso.

Após o procedimento de filtragem, foi feita uma seleção das características, ou colunas, mais relevantes para explicar o fenômeno da evasão escolar. Características como sexo, cor ou raça, se já trabalhou ou estagiou por pelo menos 1 hora em alguma atividade remunerada em dinheiro, remunerada em mercadorias e bens, não remunerada ou atividade ocasional ("bico"), número de componentes do domicílio (exclusive as pessoas cuja condição no domicílio era pensionista, empregado doméstico ou parente do empregado doméstico), se recebe bolsa família ou outro auxílio governamental e rendimento domiciliar per capita
(habitual de todos os trabalhos e efetivo de outras fontes), as quais são algumas das características que mais se correlacionam qualitativamente ao abandono e à evasão escolar no ensino médio (FERREIRA; OLIVEIRA, 2020).

Foi gerado um arquivo CSV com os dados já filtrados e características selecionadas, e posteriormente este foi lido em Python para limpeza de dados faltantes, e verificação dos tipos de variáveis com o objetivo de utilizá-lo como base de dados complementar à análise de evasão. Foi também criada a coluna "evasao" codificada de forma binária para classificar o aluno como evasão (1) e conclusão (0).


\subsection{Pré-processamento dos Dados}
As taxas obtidas a partir da base do INEP, inicialmente no formato 0 a 100 foram transformadas para formato decimal (0 a 1) como:

\vspace{0.5cm}
\begin{equation}
\large T_k = \frac{t_k}{100}
\end{equation}
\vspace{0.5cm}

Em que $T_k$ representa o vetor modificado e $t_k$ representa o vetor (coluna) da taxa original

Baseado nas limitações de tipo do modelo e da biblioteca utilizada, variáveis categóricas foram separadas em novas colunas de acordo com o número de valores únicos de texto presentes na coluna única original, transformadas cada coluna nova em vetores binários e na sequência a remoção de uma ou mais colunas para evitar correlação linear entre as características artificialmente separadas.  

Variáveis ordinais numéricas foram tranformadas em variáveis categóricas textuais e, em sequência, o fluxo de transformação citado anteriormente para variáveis categóricas foi aplicado.

Células vazias ou nulas tiveram suas linhas removidas por completo da base de modelagem para integridade da análise estatística.

\subsection{Análise de Multicolinearidade}
Multicolinearidade pode ser compreendida como uma relação linear entre duas ou mais variáveis dependentes. Conforme Paul (2006), quando há importância na investigação dos impactos dos regressores na variável dependente, a multicolinearidade pode ser um problema, visto que p-valores podem se mostrar equivocadamente elevados e em alguns casos pode interferir na interpretação dos coeficientes. Uma das maneiras de analisar se há multicolinearidade no conjunto de dados é calcular o fator de inflação de variância (\textit{VIF}), baseado em $R^2$, que indica o quanto uma variável independente é explicada pelos demais regressores para cada umas das $i$ variáveis indepedentes (MILOCA \& CONEJO, 2008).

No presente trabalho será aplicado o \textit{VIF} para verificar a mutlicolinearidade entre as variáveis advindas das bases do INEP:

\vspace{0.5cm}
\begin{equation}
\large F_i = \frac{1}{1-R_i^2}
\end{equation}
\vspace{0.5cm}


$F_i$ se refere ao fator de inflação de variância do i-ésimo regressor; $R_i^2$ se refere ao $R^2$ da i-ésima variável independente.

Um \textit{VIF} maior que 10 indica que a multicolinearidade influencia fortemente o valor dos coeficientes do modelo, como proposto por Johnson e Wichern (1988; apud MILOCA \& CONEJO, 2008), e algumas medidas em relação ao conjunto de dados ou ao modelo devem ser tomadas com o objetivo de preservar a interpretação dos dados na modelagem estatística.

\subsection{Modelo Estatístico}
Este artigo visa aplicar métodos estatísticos com o intuito de investigar o impacto das variáveis socioeconômicas e institucionais na evasão escolar e em variáveis que corroboram tal fenômeno. Para tanto, conforme Green et al (2011), que salientam a regressão linear como ferramenta amplamente usada para examinar as relações estatísticas entre variáveis, nesta pesquisa será utilizado, primordialmente, o modelo de regressão linear múltipla, expresso por:

\vspace{0.5cm}
\begin{equation}
\large Y = \beta_0 + \beta_1X_1 + \beta_2X_2 + \beta_3X_3 + \cdots + \beta_iX_i + \epsilon 
\end{equation}
\vspace{0.5cm}

Em que $Y$ é a variável dependente a ser modelada, $X_i$ são as variáveis independentes, ou regressores, $\beta_i$ representa os coeficientes atrelados a cada variável independente e $\epsilon$ representa o erro aleatório. 

Com o modelo, e o auxílio da técnica dos mínimos quadrados para quantificar os coeficientes, é possível mensurar a significância das diferentes variáveis independentes, e o quanto sua variação impacta na variável dependente, de acordo com o contexto das amostras e das variáveis coletadas.



\subsection{Métricas de Desempenho}
Com o objetivo de extrair resultados acerca do desempenho do modelo deste trabalho, foram utilizadas as técnicas $R^2$ e a média dos quadrados dos resíduos (\textit{MSE}). Por meio destas é possível ter ciência do quanto o modelo explica a variação da variável dependente que varia de 0 a 1 (sendo 1 o ajuste perfeito dos dados pelo modelo) e a média da magnitude de erros do modelo, respectivamente:

\vspace{0.5cm}
\begin{equation}
\large R^2 = 1 - \frac{\sum_i(y_i-\hat{y}_i)^2}{\sum_i(y_i-\bar{y})^2}
\end{equation}
\vspace{0.5cm}

\vspace{0.5cm}
\begin{equation}
\large E_{qm} = \frac{1}{n} \sum_i(y_i-\hat{y_i})^2
\end{equation}
\vspace{0.5cm}

De forma que $y_i$ é o valor real, $\hat{y_i}$ é valor predito pelo modelo e $\bar{y}$ é a média dos valores da variável dependente.

\section{Resultados Parciais}

Com base nos dados coletados a partir dos dados abertos do INEP e na regressão linear múltipla resumida na tabela 1:

\begin{table}[htbp]
\centering
\caption{Resultados da regressão linear MQO para a variável dependente \texttt{TX\_EV\_TOTAL}}
\label{tab:regressao_tx_ev_sem_ic}
\begin{tabular}{lrrrr}
\hline
\textbf{Variável} & \textbf{Coef.} & \textbf{Erro Padrão} & \textbf{t} & \textbf{P$>$|t|} \\
\hline
const & -0,0029 & 0,003 & -0,898 & 0,369 \\
DEP\_ADMINISTRATIVA\_Pública & 0,0273 & 0,002 & 16,393 & 0,000 \\
Grupo 2 & 0,0572 & 0,009 & 6,338 & 0,000 \\
Grupo 3 & -0,0115 & 0,004 & -3,189 & 0,001 \\
Grupo 4 & -0,0168 & 0,006 & -2,989 & 0,003 \\
Grupo 5 & 0,0049 & 0,005 & 1,079 & 0,281 \\
IN\_BIBLIOTECA & 0,0016 & 0,002 & 0,694 & 0,488 \\
IN\_COZINHA & -0,0031 & 0,003 & -1,143 & 0,253 \\
IN\_INTERNET & 0,0178 & 0,003 & 6,740 & 0,000 \\
IN\_LABORATORIO\_CIENCIAS & -0,0052 & 0,004 & -1,253 & 0,210 \\
IN\_LABORATORIO\_INFORMATICA & 0,0011 & 0,003 & 0,413 & 0,680 \\
IN\_QUADRA\_ESPORTES & 0,0025 & 0,003 & 0,881 & 0,378 \\
NUM\_ALUNO\_TURMA\_TOTAL & 0,0000 & 0,000 & 0,642 & 0,521 \\
TX\_DI\_TOTAL & 0,1853 & 0,004 & 41,654 & 0,000 \\
TX\_IED\_N5 & 0,0070 & 0,004 & 1,767 & 0,077 \\
TX\_IED\_N6 & -0,0112 & 0,006 & -1,939 & 0,053 \\
TX\_REP\_TOTAL & 0,0528 & 0,009 & 5,622 & 0,000 \\
\hline
\multicolumn{5}{l}{\footnotesize $R^2 = 0,514$, Estatística-F = 455,7, Prob(F) = 0,00, $E_{qm}=0,0026$} \\
\multicolumn{5}{l}{\footnotesize Observações = 6915} \\
\end{tabular}
\end{table}

O modelo tem erro quadrático médio $E_{qm}=0,0026$ o que indica que erra cerca de 5 pontos percentuais para fins de predição. Ao mesmo tempo, apresenta um $R^2=0,514$, explicando $51\%$ da variação dos dados, e uma $Prob(F)=0$, que indica que o modelo é significativo para explicar a variável dependente.

Os resultados mostram que, ao manter as outras variáveis constantes, quando TX\_DI\_TOTAL, a taxa de defasagem idade-série, aumenta em 1 ponto percentual, a taxa de evasão total tende a aumentar em 0,1853 pontos percentuais. O coeficiente é estatisticamente significativo com $p<0.001$ e se mostra como o coeficiente mais alto que se relaciona com a variável dependente TX\_EV\_TOTAL.

De maneira semelhante, TX\_REP\_TOTAL, que demonstra a taxa de repetência total do município, também apresenta alto coeficiente em comparação com as outras variáveis, sendo $0,0528$, e estatisticamente significativo com $p<0,0001$.

Os resultados também sugerem que, com $N=6915$ observações, as variáveis de infraestrutura coletadas não se mostram, em sua maioria, como estatisticamente significadas, e aquela que apresentam p-valor menor que 5\% (IN\_INTERNET) exibe coeficiente $\beta_i<0,0178$, o que caracteriza uma ínfima variação de pontos percentuais na taxa de evasão total. Taxas do índice de esforço do docente (TX\_IED\_N5 e TX\_IED\_N5) de forma análoga não se mostram estatisticamente significativos, com $p>0,05$.

Em resumo, as variáveis Grupo 2, TX\_DI\_TOTAL, TX\_REP\_TOTAL, são evidenciadas como as mais relevantes no presente estudo. 

com base, em particular, na taxa de repetência, decidiu-se utilizar a base do Sistema de Avaliação do Educação básica (SAEB) do INEP, para investigar quais variáveis mais impactam na taxa de repetência e, de forma implícita, na taxa de defasagem idade-série, dado que a repetência leva à defasagem em relação à série cursada.

Com base na tabela 2, que resume os resultados da regressão linear múltipla da base de dados do SAEB:

\begin{table}[htbp]
\centering
\caption{Resultados da regressão linear MQO para a variável dependente \texttt{TX\_RESP\_Q041}}
\label{tab:regressao_tx_resp_q041}
\small
\begin{tabular}{lrrrrr}
\hline
\textbf{Variável} & \textbf{Coeficiente} & \textbf{Erro padrão} & \textbf{t} & \textbf{p$>$|t|} & \textbf{[0,025 ; 0,975]} \\
\hline
const & 0,6980 & 0,039 & 17,716 & 0,000 & [0,621 ; 0,775] \\
PROFICIENCIA\_MT & -0,1567 & 0,003 & -61,733 & 0,000 & [-0,162 ; -0,152] \\
TX\_RESP\_Q001 & 0,0841 & 0,010 & 8,203 & 0,000 & [0,064 ; 0,104] \\
TX\_RESP\_Q002\_Indígena & 0,1083 & 0,018 & 5,879 & 0,000 & [0,072 ; 0,144] \\
TX\_RESP\_Q002\_Parda & 0,0474 & 0,007 & 6,716 & 0,000 & [0,034 ; 0,061] \\
TX\_RESP\_Q002\_Preta & 0,1422 & 0,012 & 11,985 & 0,000 & [0,119 ; 0,165] \\
TX\_RESP\_Q027 & -0,4332 & 0,039 & -11,121 & 0,000 & [-0,510 ; -0,357] \\
TX\_RESP\_Q038 & 0,0595 & 0,007 & 8,035 & 0,000 & [0,045 ; 0,074] \\
TX\_RESP\_Q044 & -0,1006 & 0,010 & -10,433 & 0,000 & [-0,120 ; -0,082] \\
TX\_RESP\_Q052 & 0,0965 & 0,009 & 11,240 & 0,000 & [0,080 ; 0,113] \\
\hline
\multicolumn{6}{l}{\footnotesize $R^2 = 0,272$, Estatística F = 782,4, Prob(F) = 0,00, $E_{qm}=0,0299$} \\
\multicolumn{6}{l}{\footnotesize Número de observações = 18846} \\
\end{tabular}
\end{table}

O modelo se mostra estatisticamente significativo dada $Prob(F)=0$, e explica 27,2\% da variação dos dados coletados.

Pode-se observar que das variáveis coletadas todas são significativas com $p<0,001$, e se destacam as variáveis TX\_RESP\_Q027, taxa de alunos incentivados pelos pais aos estudos, com um alto coeficiente de $\beta_i=-0,4332$, PROFICIENCIA\_MT ($\beta_i=-0,1567$), que mede a proficiência de matemática média da instituição; TX\_RESP\_Q002\_Preta ($\beta_i=0,1422$), que refere-se à taxa de alunos autodeclarados pretos na instituição de ensino; TX\_RESP\_Q002\_Indígena ($\beta_i=0,1083$), que faz jus à taxa de alunos autodeclarados indígenas. Tais variáveis possuem os maiores coeficientes do conjunto de regressores. 

De acordo com os resultados, o aumento de 1 ponto percentual na taxa de alunos incentivados ao estudo está associado à diminuição de 0,4332 pontos percentuais na taxa de repetência. De modo semelhante, o aumento de 1 ponto percentual na proficiência em matemática tem associação com a diminuição em 0,1567 pontos percentuais da variável dependente.

Por outro lado, o aumento de 1 ponto percentual no grupo de alunos autodeclarados pretos nas instituições tende a elevar a taxa de repetência em 0,1422. E o mesmo pode ser observado para o grupo de alunos indígenas, cujo aumento de 1 ponto percentual implica em $+0.1083$ pontos percentuais na taxa de repetência. Ademais, o aumento de 1 ponto percentual na taxa de alunos que trabalham (TX\_RESP\_Q038) está associado a $+0,0595$ pontos percentuais na taxa de repetência.

A variável TX\_RESP\_Q044, que faz referência à taxa de alunos que gostam de estudar língua portuguesa, também mostra que com o aumento de 1 ponto percentual, a taxa de repetência tende a $-0,1006$ pontos percentuais. E curiosamente, o aumento de 1 ponto percentual na taxa de alunos que demonstram interesse por matemática (TX\_RESP\_Q052) está associado a um aumento de aproximandamente 0,09 pontos percentuais na taxa de repetência.

A correlação entre evasão, repetência e defasagem idade-série é corroborada pela cenário do ensino médio brasileiro. De acordo com dados da PNAD, pesquisa nacional realizada pelo IBGE, coletados nesta pesquisa para análise complementar, a taxa de evasão entre pretos em idade escolar adequada ao ensino secundário é de 22,87\%, entre pardos 22,26\%, e entre indígenas equivalente a 25,64\% (com base na lógica de cálculo de evasão deste trabalho). Brancos apresentam o menor número com cerca de 16,16\%. 

No índice de repetência, o grupo de pretos e pardos é cerca de 2 a 4 pontos percentuais acima do percentual de brancos que se mantêm na mesma série em todo o Brasil, de acordo com dados do INEP. Ainda segundo a plataforma inepdata, a taxa de jovens pretos e pardos que estão defasados no quesito idade-série, chega a ser de 10 pontos percentuais em comparação com jovens brancos.

A evasão e dificuldade escolar de tais grupos fragilizados também são refletidos em suas condições socioeconômicas e o inverso também é verdadeiro. Jovens pretos em idade escolar que já estão fora do ensino médio devido à conclusão ou evasão apresentam renda média mensal de 790 reais, jovens pardos 780 reais, jovens indígenas 727 reais, e jovens brancos 1230 reais em média, segundo dados da PNAD no contexto desta pesquisa.

Ante o exposto, os resultados parciais reforçam que, em alguma medida, fatores de infraestutura e institucionais não são signficativos para explicar a evasão e o índice de repetência tal como fatores acadêmicos e sociais, como a raça e o incentivo doméstico aos estudos e manutenção de presença na instituição de ensino.

\subsection{Descrição de Variáveis Selecionadas}
Para analisar estatisticamente a evasão e a repetência, foram selecionadas as seguintes variáveis descritas nas tabelas 3 (regressão da taxa de evasão como variável dependente) e 4 (regressão com taxa de repetência como variável dependente):

\begin{table}[htbp]
\centering
\caption{Variáveis explicativas do modelo de regressão da taxa de evasão total (\texttt{TX\_EV\_TOTAL})}
\label{tab:variaveis_tx_ev}
\small
\begin{tabular}{lp{9cm}}
\hline
\textbf{Variável} & \textbf{Descrição} \\
\hline
DEP\_ADMINISTRATIVA\_Pública & Escola pública (1 = sim, 0 = não) \\
Grupo 2 & Docentes com bacharelado na mesma área (Grupo 2) \\
Grupo 3 & Docentes com licenciatura ou bacharelado em área diferente (Grupo 3) \\
Grupo 4 & Docentes com formação superior não enquadrada (Grupo 4) \\
Grupo 5 & Docentes sem formação superior (Grupo 5) \\
IN\_BIBLIOTECA & Percentual de escolas do município com biblioteca \\
IN\_COZINHA & Percentual de escolas do município com cozinha \\
IN\_INTERNET & Percentual de escolas do município com acesso à Internet \\
IN\_LABORATORIO\_CIENCIAS & Percentual de escolas do município com laboratório de ciências \\
IN\_LABORATORIO\_INFORMATICA & Percentual de escolas do município com laboratório de informática \\
IN\_QUADRA\_ESPORTES & Percentual de escolas do município com quadra de esportes \\
NUM\_ALUNO\_TURMA\_TOTAL & Número médio de alunos por turma \\
TX\_DI\_TOTAL & Taxa de distorção idade-série \\
TX\_IED\_N5 & Percentual de docentes nível 5 do Índice de Esforço Docente \\
TX\_IED\_N6 & Percentual de docentes nível 6 do Índice de Esforço Docente \\
TX\_REP\_TOTAL & Taxa total de reprovação \\
\hline
\end{tabular}
\\[2mm]
\footnotesize Fonte: Elaborado pelo autor a partir do Censo Escolar e do INEP.
\end{table}

\begin{table}[htbp]
\centering
\caption{Variáveis da regressão da taxa de repetência (\texttt{TX\_RESP\_Q041})}
\label{tab:vars_tx_resp_q041}
\small
\begin{tabular}{lp{10cm}}
\hline
\textbf{Variável} & \textbf{Descrição} \\
\hline
PROFICIENCIA\_MT & Proficiência média em Matemática da escola \\
TX\_RESP\_Q001 & Percentual de alunos do sexo masculino \\
TX\_RESP\_Q027 & Percentual de alunos que são incentivados pelos pais ao estudo \\
TX\_RESP\_Q002\_Indígena & Percentual de alunos autodeclarados indígenas \\
TX\_RESP\_Q002\_Parda & Percentual de alunos autodeclarados pardos \\
TX\_RESP\_Q002\_Preta & Percentual de alunos autodeclarados pretos \\
TX\_RESP\_Q038 & Percentual de alunos que trabalham fora de casa atualmente \\
TX\_RESP\_Q044 & Percentual de alunos que gostam de estudar Língua Portuguesa \\
TX\_RESP\_Q052 & Percentual de alunos que gostam de estudar Matemática \\
\hline
\end{tabular}
\\[2mm]
\footnotesize\centering Fonte: Elaborado pelo autor a partir da base SAEB do INEP.
\end{table}




















%====================================================================


%See the guidelines for metadata and references:
%https://sol.sbc.org.br/journals/index.php/rbie/libraryFiles/downloadPublic/71
%====================================================================

\pagebreak
\pagebreak
\nocite{*}
\printbibliography


\end{document}
