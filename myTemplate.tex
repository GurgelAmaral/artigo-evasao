%
% Template for RBIE papers in LaTeX
%

% The above language combination is for this template document only.
% You should use one of the following:
\documentclass[english, spanish, brazilian]{RBIEarticle} % for papers in portuguese
%\documentclass[brazilian, spanish, english]{RBIEarticle} % for papers in english
%\documentclass[brazilian, english, spanish]{RBIEarticle} % for papers in spanish

% Papers in Portuguese or Spanish may require the following lines:
\usepackage[utf8]{inputenc} % chooses UTF-8 as the main character set
\usepackage[T1]{fontenc} % for correct syllable separation in accented words
% Pacotes para citações/referências ABNT
%usepackage[alf]{abntex2cite} % citações autor-data
%\usepackage[num]{abntex2cite} % citações numéricas
\usepackage{amsmath}

% The next two statements are needed for the example table in this document
% (i.e. you don't necessarily need them in your own paper)
\usepackage{colortbl}
\definecolor{gray}{gray}{.8}

% Citations and references (Biblatex)
% Citations and references (Biblatex)
\usepackage[style=abnt]{biblatex}
\usepackage{csquotes}
\addbibresource{references.bib}

% Here goes the paper main title
\title{Modelagem Estatística da Evasão no Ensino Médio Brasileiro: Uma Investigação Baseada em Variáveis Institucionais e Socioeconômicas}

% If the manuscript is written in English, then this element must be removed.
\titleinenglish{Statistical Modeling of Dropout Rates in Brazilian High Schools: An Investigation Based on Institutional and Socioeconomic Variables}

% If the manuscript is written in English, then this element must be removed.
\titleinspanish{Modelización estadística de la deserción escolar en la enseñanza secundaria brasileña: una investigación basada en variables institucionales y socioeconómicas}

% Here goes the paper author information (repeat for two or more authors)
\author{%
	\parbox{3.8cm}{%
		Bruno Alexandre Dias da Silva\\
		Universidade de São Paulo\\
		ORCID: \href{https://orcid.org/0000-0000-0000-0000}{0000-0000-0000-0000}\\
		Brunoalexdias20@usp.br
	}
        \hspace{0.3cm}
	\parbox{3.8cm}{%
		Lucas Gurgel do Amaral\\
		Universidade de São Paulo\\
		ORCID: \href{https://orcid.org/0000-0000-0000-0000}{0000-0000-0000-0000}\\
		lucasgurgel@usp.br
	}
        \hspace{0.3cm}
        \parbox{3.8cm}{%
		Rafael de França\\
		Universidade de São Paulo\\
		ORCID: \href{https://orcid.org/0000-0000-0000-0000}{0000-0000-0000-0000}\\
		rafaeldefranca@usp.br
	}
        \hspace{0.3cm}
	\parbox{3.9cm}{\raggedright%
		Richard Pereira do Nascimento\\
		Universidade de São Paulo\\
		ORCID: \href{https://orcid.org/0000-0000-0000-0000}{0000-0000-0000-0000}\\
		rcdwoods@usp.br
	}
}

\Submission{dd/Mmm/yyyy}
\First_round_notif{dd/Mmm/yyyy}
\New_version{dd/Mmm/yyyy}
\Second_round_notif{dd/Mmm/yyyy}
\Camera_ready{dd/Mmm/yyyy}
\Edition_review{dd/Mmm/yyyy}
\Available_online{dd/Mmm/yyyy}
\Published{dd/Mmm/yyyy}

% Here goes the page heading information
\heading{Gurgel et al.
}{RBIE v.VV – 2025}

% And finally here goes the citation information
\citeas{Last name, Initials., \ldots \& Last name, Initials.  (Year). Article title in the original language. Revista Brasileira de Informática na Educação, vol, pp-pp. https://doi.org/10.5753/rbie.yyyy.id}

\citeas{
SILVA, B. A. D.; GURGEL, L.; FRANÇA, R. D.; NASCIMENTO, R. P. do. Aprendizado de Máquina Aplicado à Predição de Evasão no Ensino Médio em São Paulo. Revista Brasileira de Informática na Educação, vol , pp-pp, 2025.
}

%====================================================================
%\hyphenpenalty=10000
%\setcounter{page}{01}

\begin{document}
\maketitle

% If the manuscript is written in English, then this element must be removed.
\begin{otherlanguage}{brazilian}
\begin{abstract}
A evasão escolar no ensino brasileiro vem se mostrando como um grande desafio na qualificação e formação educacional do público infanto-juvenil, principalmente nas camadas menos favorecidas da sociedade brasileira. Jovens que não concluem o ensino médio não conseguem especializar-se em cursos superiores e, portanto, submetem-se a empregos com mão de obra barata, perdurando um ciclo de pobreza e baixos indicadores socioeconômicos. Para isso, este trabalho tem como objetivo avaliar e desenvolver modelos probabilísticos de aprendizado de máquina com o intuito de prever a probabilidade de um aluno evadir o ensino médio da rede de ensino de São Paulo baseado em técnicas como Floresta Aleatória, Árvores de Decisão e Redes Neurais. Por meio da base de dados da Pesquisa Nacional por Amostra de Domicílios (PNAD) provenientes do Instituto Brasileiro de Geografia e Estatística (IBGE), serão extraídas variáveis que, de acordo com a literatura de estudos acerca de evasão, mais expliquem o abandono escolar contínuo. Os resultados deste artigo podem auxiliar docentes e escolas a rastrearem e prestarem mais apoio àqueles alunos que apresentam alta chance de evasão por motivos socioeconômicos.  
\keywords\ Evasão escolar; Aprendizado de Máquina; PNAD; Variáveis Socioeconômicas; Ensino médio.
\end{abstract}
\end{otherlanguage}

\begin{otherlanguage}{english}
\begin{abstract}
School dropout rates in Brazil have proven to be a major challenge in the educational qualification and training of children and young people, especially in the less privileged segments of Brazilian society. Young people who do not complete high school are unable to specialize in higher education courses and, therefore, take on low-wage jobs, perpetuating a cycle of poverty and low socioeconomic indicators. Therefore, this study aims to evaluate and develop probabilistic machine learning models to predict the probability of a student dropping out of high school in the São Paulo school system based on techniques such as Random Forest, Decision Trees, and Neural Networks. Using data from the Brazilian Institute of Geography and Statistics (IBGE) National Household Sample Survey (PNAD), variables will be extracted that, according to the literature on dropout, best explain continuous school abandonment. The results of this article can help teachers and schools track and provide more support to those students who are at high risk of dropping out for socioeconomic reasons.
\keywords\ School dropout; Machine learning; PNAD; Socioeconomic variables; Secondary education.
\end{abstract}
\end{otherlanguage}

% If the manuscript is written in English, then this element must be removed.
\begin{otherlanguage}{spanish}
\begin{abstract}
La deserción escolar en la educación brasileña se ha convertido en un gran desafío para la calificación y la formación educativa de los niños y jóvenes, especialmente en los sectores menos favorecidos de la sociedad brasileña. Los jóvenes que no completan la educación secundaria no pueden especializarse en cursos superiores y, por lo tanto, se ven obligados a aceptar empleos con mano de obra barata, perpetuando un ciclo de pobreza y bajos indicadores socioeconómicos. Por ello, este trabajo tiene como objetivo evaluar y desarrollar modelos probabilísticos de aprendizaje automático con el fin de predecir la probabilidad de que un estudiante abandone la educación secundaria en la red educativa de São Paulo, basándose en técnicas como bosques aleatorios, árboles de decisión y redes neuronales. A partir de la base de datos de la Encuesta Nacional por Muestra de Hogares (PNAD) del Instituto Brasileño de Geografía y Estadística (IBGE), se extraerán variables que, según la literatura de estudios sobre el abandono escolar, explican mejor el abandono escolar continuo. Los resultados de este artículo pueden ayudar a los docentes y a las escuelas a rastrear y brindar más apoyo a aquellos alumnos que presentan un alto riesgo de deserción por motivos socioeconómicos.
\keywords\ Absentismo escolar; Aprendizaje automático; PNAD; Variables socioeconómicas; Educación secundaria.
\end{abstract}
\end{otherlanguage}

\pagebreak

%====================================================================

\section{Introdução}

A evasão escolar é definida como a ausência de retorno ao sistema de ensino formal do aluno em idade escolar após abandono ou reprovação. Deste modo, enquanto o abandono faz referência à situação do aluno que deixa de frequentar as aulas durante o ano letivo, a evasão refere-se ao aluno que, por qualquer motivo, não regressou à rede de ensino com o reinício do ano letivo, assim como descrito pelo manual guia do Sistema de Alerta Preventivo de Evasão e Abandono Escolar (SAP).

Elucidada sua definição, a evasão se mostra um grande desafio na esfera do ensino brasileiro, em especial ao ensino público. Segundo os indicadores de taxa de transição e fluxo provenientes do Censo Escolar, em nível nacional e na etapa do ensino médio, a taxa de evasão apresenta valores preocupantes: enquanto a rede privada registra índices de aproximadamente 2\% a 3\% na série histórica, a rede pública chega a patamares entre 7\% e 12\% do contingente total de alunos (INEP, 2022). No Brasil como um todo, de acordo com dados do IBGE (2021), cerca de 5 milhões de jovens entre 14 e 29 anos estavam fora da escola, sendo que uma parcela significativa não havia concluído o ensino médio.

A literatura especializada evidencia que o fenômeno da evasão é multifatorial. Silva (2013) ressalta que o nível socioeconômico das famílias é um dos principais fatores que impele estudantes a não retomar a vida acadêmica. Um baixo nível socioeconômico, que reflete também a pobreza, induz indivíduos a abandonar os estudos à procura de trabalho para complementar a renda familiar mensal. A dificuldade em conciliar estudo e trabalho, a necessidade de contribuir economicamente para o domicílio e a falta de suporte institucional figuram como causas recorrentes. Além disso, a ausência de políticas públicas eficazes de acompanhamento e prevenção agrava o problema, pois não há um sistema consolidado de rastreamento que permita identificar estudantes em risco de evasão.

Do ponto de vista das consequências sociais, a evasão escolar impacta diretamente a qualificação da mão de obra e perpetua ciclos de pobreza. Jovens que não concluem o ensino médio enfrentam maiores barreiras de acesso ao ensino superior e ficam restritos a ocupações de baixa remuneração e instabilidade, reduzindo suas chances de mobilidade social. Segundo o IBGE (2021), a renda média de um trabalhador com ensino médio completo pode ser até 40\% superior à de um trabalhador que abandonou os estudos antes de concluir essa etapa. Dessa forma, o fenômeno não afeta apenas o indivíduo, mas também repercute nos indicadores socioeconômicos do país, ampliando desigualdades estruturais.

Comparações internacionais reforçam a gravidade do cenário brasileiro. Dados da UNESCO (2020) mostram que o Brasil apresenta taxas de abandono escolar superiores à média de países da América Latina e muito distantes das registradas em nações da OCDE, onde os índices de evasão no ensino médio são inferiores a 5\%. Essa discrepância aponta para a necessidade de maior investimento em políticas educacionais de permanência, associadas ao monitoramento de fatores de risco.

Diante desse panorama, estudos que integrem variáveis socioeconômicas e técnicas computacionais de análise se mostram fundamentais. A Pesquisa Nacional por Amostra de Domicílios (PNAD), realizada pelo IBGE, constitui uma das principais fontes de dados para compreender o perfil dos estudantes e identificar as condições de vulnerabilidade associadas ao abandono escolar. Ao possibilitar a análise de fatores como renda domiciliar, escolaridade dos responsáveis, ocupação e acesso a programas sociais, a PNAD oferece subsídios robustos para a construção de modelos preditivos.

Nesse sentido, este artigo investiga a problemática da evasão escolar no Brasil a partir da PNAD, com a finalidade de analisar e desenvolver modelos probabilísticos baseados em aprendizado de máquina. O objetivo é avaliar a probabilidade de evasão escolar de alunos com base em suas características socioeconômicas, a fim de proporcionar um método de predição capaz de auxiliar gestores educacionais na implementação de estratégias preventivas e compreender de que forma o perfil do estudante impacta seu desenvolvimento acadêmico e social.



\section{Fundamentos Teóricos}

O presente capítulo tem como objetivo apresentar a base conceitual que sustenta esta pesquisa, fornecendo o embasamento necessário para compreender as abordagens utilizadas na predição da evasão escolar no Ensino Médio. A fundamentação teórica é fundamental para situar o estudo no contexto das pesquisas já existentes, permitindo identificar conceitos-chave, modelos consolidados e contribuições de autores que exploraram temas relacionados à evasão escolar e ao uso de técnicas de aprendizado de máquina em contextos educacionais.

\subsection{Evasão Escolar e Fatores Socioeconômicos}
A evasão escolar é um fenômeno amplamente estudado nas áreas da Educação e das Ciências Sociais, podendo ser associado a fatores de ordem social, econômica e cultural. Segundo Araque, Roldán e Salguero (2009), as principais razões que influenciam o abandono da escola estão relacionadas a variáveis socioeconômicas, familiares, institucionais e psicológicas. Nesse sentido, compreender esse fenômeno requer considerar tanto o contexto social do aluno quanto as condições oferecidas pelo sistema educacional.

\subsection{Mineração de Dados Educacionais e Aprendizado de Máquina}
No campo da análise de dados, o termo Mineração de Dados Educacionais (\textit{Educational Data Mining — EDM}) refere-se ao uso de métodos computacionais para explorar grandes volumes de dados gerados em ambientes educacionais, buscando padrões relevantes (BAKER; ISOTANI; CARVALHO, 2011). O objetivo disso é apoiar a tomada de decisão em políticas públicas e institucionais, fornecendo evidências que permitam compreender e reduzir problemas como a própria evasão escolar.

Associado à EDM, o Aprendizado de Máquina (\textit{Machine Learning}) é um subcampo da Inteligência Artificial que possibilita a criação de modelos preditivos a partir de dados. Esses modelos são capazes de identificar relações complexas entre variáveis e gerar previsões com base em informações históricas (MITCHELL, 1997). No contexto da evasão escolar, técnicas de aprendizado supervisionado permitem estimar a probabilidade de um estudante abandonar ou concluir seus estudos, dadas suas características socioeconômicas, acadêmicas e pessoais (TEODORO; KAPPEL, 2020).

\subsection{Principais Técnicas Utilizadas}
Entre as abordagens mais comuns em problemas de classificação binária, destacam-se a Regressão Logística, as Redes Neurais e os métodos de Floresta Aleatória de Classificação. A Regressão Logística (RADÜNZ, 1992) permite modelar a relação entre variáveis independentes e uma variável dependente categórica, atribuindo probabilidades ao evento de interesse. Já as Redes Neurais, inspiradas no funcionamento do cérebro humano, são capazes de identificar padrões não-lineares e de alta complexidade (GARDNER; DORLING, 1998). Por fim, o método de Florestas Aleatórias, proposto por Breiman (2001), combina múltiplas árvores de decisão para melhorar a acurácia preditiva e reduzir problemas de sobreajuste (\textit{overfitting}).

\subsection{Técnicas Computacionais Complementares}
Embora este trabalho foque na aplicação de Regressão Logística, Redes Neurais e Florestas Aleatórias de Classificação, outras técnicas computacionais também têm sido amplamente empregadas em estudos de predição e análise de dados educacionais. Entre elas, destacam-se:

\textbf{Máquinas de Vetores de Suporte (SVM)}: modelo supervisionado que busca encontrar um hiperplano ótimo de separação entre classes. As SVMs são reconhecidas por sua eficiência em problemas de classificação binária e por sua capacidade de lidar com dados de alta dimensionalidade, sendo frequentemente utilizadas em contextos educacionais para a identificação de padrões de desempenho acadêmico (CORTES; VAPNIK, 1995).

\textbf{Naive Bayes}: técnica baseada em probabilidade que utiliza o Teorema de Bayes para estimar a classe de uma observação com base na distribuição condicional das variáveis independentes. Apesar de sua simplicidade e da suposição de independência entre atributos, o Naive Bayes apresenta bons resultados em problemas de classificação de texto e dados categóricos, podendo ser aplicado à análise de questionários e registros escolares (ZHANG, 2004).

\textbf{K-Nearest Neighbors (KNN)}: algoritmo que classifica uma observação com base na proximidade de seus vizinhos mais próximos em um espaço multidimensional. O KNN é intuitivo e eficaz em situações em que não se pressupõe uma relação linear entre as variáveis, sendo útil para identificar perfis de estudantes com risco de evasão a partir de características socioeconômicas semelhantes (COVER; HART, 1967).

Além dessas técnicas, também se destacam os métodos de \textit{clustering} como o K-Means, utilizados em contextos de aprendizado não supervisionado. Esses métodos permitem segmentar grupos de estudantes com características semelhantes, favorecendo análises exploratórias sobre padrões de evasão sem a necessidade de rótulos previamente definidos.

\subsection{A Pesquisa Nacional por Amostra de Domicílios (PNAD)}
A Pesquisa Nacional por Amostra de Domicílios (PNAD), realizada pelo Instituto Brasileiro de Geografia e Estatística (IBGE), constitui uma das principais fontes de dados socioeconômicos no Brasil. Seu objetivo é coletar informações abrangentes sobre características demográficas, educacionais, ocupacionais e de rendimento da população brasileira, por meio de entrevistas domiciliares aplicadas em amostras representativas em nível nacional, regional e estadual (IBGE, 2022).

A PNAD Contínua, em vigor desde 2012, aprimorou o levantamento ao adotar coleta trimestral, permitindo análises mais atualizadas e consistentes acerca da dinâmica social e econômica do país. Entre suas variáveis, destacam-se renda familiar per capita, inserção no mercado de trabalho, características do domicílio, composição familiar, escolaridade e acesso a programas sociais. Tais informações são de grande relevância para estudos sobre evasão escolar, pois possibilitam identificar relações entre vulnerabilidade socioeconômica e permanência na escola.

\subsection{Importância da PNAD no Contexto Acadêmico}
A PNAD desempenha papel fundamental como fonte de dados em pesquisas acadêmicas brasileiras, especialmente nas áreas de Ciências Sociais, Economia e Educação. Por sua abrangência e representatividade estatística, a pesquisa possibilita análises robustas sobre condições de vida, desigualdade e mobilidade social. Estudos desenvolvidos por instituições de ensino e pesquisa frequentemente utilizam a PNAD como base empírica para compreender fenômenos relacionados ao trabalho, à renda e à escolaridade da população.

No campo educacional, a PNAD se destaca por permitir a investigação de variáveis socioeconômicas associadas à permanência ou evasão escolar. A inclusão de indicadores como nível de escolaridade dos responsáveis, rendimento domiciliar per capita, participação no mercado de trabalho e acesso a programas sociais fornece subsídios para estudos que buscam identificar fatores de risco e vulnerabilidade entre os estudantes. Adicionalmente, a consolidação histórica da PNAD garante comparabilidade ao longo do tempo, permitindo a construção de séries históricas sobre aspectos socioeconômicos.

\subsection{Síntese}
Dessa forma, os fundamentos apresentados evidenciam a relevância de associar teorias sobre evasão escolar com métodos de mineração de dados e aprendizado de máquina, bem como o uso de bases de dados socioeconômicas amplas como a PNAD. Essa integração permite analisar grandes quantidades de dados e construir modelos preditivos capazes de apoiar políticas públicas e institucionais na área da Educação, contribuindo para a redução da evasão escolar e para a promoção da equidade educacional.

\section{Trabalhos Relacionados}
Diversos estudos buscaram modelar de forma probabilística ou com técnicas de aprendizado de máquina o fenômeno da evasão escolar no Brasil. Como exemplo, destacam dois trabalhos acerca da evasão no âmbito acadêmico.

Em Mello et al. (2023) foram realizadas análises exploratórias acerca das características consideradas (variáveis independentes) como coeficiente de rendimento, tipo de escola de origem, percentual de frequência, nível de ensino, cor ou raça, modalidade de curso, sexo, renda familiar bruta, situação acadêmica, dentre outros aspectos, de alunos do Instituto Federal do Pernambuco (IFPE). Com a análise constatou-se que, como alguns exemplos, alguns cursos e períodos apresentam evasão maior que outros; o percentual de evasão do sexo masculino era maior em comparação ao feminino. Contudo, não houve análise acerca da renda per capita dos indivíduos, que é reiterada por autores que exploraram o tema de forma qualitativa, como Silva (2013) e Ferreira \& Oliveira (2020). Os autores utilizaram como técnica de aprendizado de máquina para a previsão de evasão XGBoost, Florestas Aleatórias e Árvores de decisão, para predizer a situação de matrícula do estudante conforme a base de dados disponível, e obtiveram, respectivamente, 82\%, 83\% e 80\% de acurácia. Esses resultados sugerem que tais modelos apresentam ótimo desempenho preditivo para estimar as probabilidades de evasão escolar, em concordância com o objetivo deste trabalho.

Por outro lado, o trabalho de Teodoro e Kappel (2020) também segue a mesma lógica do trabalho supracitado, com a ressalva de que tem ênfase em atributos diferentes e apresenta uma análise exploratória mais robusta acerca das características correlacionadas à evasão. Teodoro e Kappel optaram por desenvolver modelos preditivos baseados em Naive Bayes, KNN, Árvores de Decisão, Florestas Aleatórias de Classificação e Redes Neurais, com acurácias de 60\%, 75\%, 77\%, 79\% e 78\%, respectivamente, no geral. A apuração mostra que, em termos de acurácia, Florestas Aleatórias e Redes Neurais são as técnicas mais adequadas para estimação de evasão. Porém, assim como o estudo anterior, tem como principal objeto de estudo a evasão no ensino superior.  

Nota-se que os estudos revisados não abordam a evasão do ensino médio nas escolas públicas. Há escassez de análise e implementação de variáveis acerca dos fatores socioeconômicos como o trabalho, a renda e a fome. Nesse sentido, o presente artigo pretende examinar e preencher estas lacunas, ao analisar e implementar atributos variados e propor modelos que expliquem de modo probabilístico o tópico.

\section{Metodologia}
O Instituto Nacional de Estudos e Pesquisas Educacionais Anísio Teixeira (INEP) dispõe uma vasta base de microdados e dados abertos baseados em variáveis institucionais, acadêmicas e socioeconômicas do sistema de ensino brasileiro no decorrer das últimas 2 décadas. Como complemento, a Pesquisa Nacional por Amostra de Domicílios (PNAD), realizada pelo Instituto Brasileiro de Geografia e Estatística (IBGE), é um conjunto de informações detalhadas sobre o cenário socioeconômico da sociedade brasileira. A partir dela, é possível extrair dados e informações das características gerais da população.  

Apoiado em tais conjuntos de dados, a metodologia deste artigo consiste no processamento desses dados e a utilização de técnicas estatísticas que possibilitem uma análise quantitativa acerca da evasão de alunos do ensino médio brasileiro, e quais fatores explicam tal fenômeno, dadas as suas características de cunho social, econômico, acadêmico e institucional. Para tal, foram utilizadas as linguagens R, Python e a biblioteca Statsmodels.  

Conjuntos de dados provenientes do INEP foram extraídos do site de dados abertos do instituto na aba de "Indicadores Educacionais" e "Microdados". Bases de dados como a taxa de repetência e de evasão, taxa de distorção idade-série, índice do esforço docente, número médio de alunos por turma, nível de adequação do docente e microdados da educação básica (todos a nível municipal e do ano de 2017) foram coletados, extraindo apenas ocorrências do ensino médio, em formato de planilhas xlsx ou arquivo com valores separados por vírgula (\textit{CSV}) e lidos posteriormente em Python com a biblioteca de manipulação de dados Pandas.

Em sequência, foi realizada a junção das bases supracitadas por meio das colunas, renomeadas a partir da amostra original, UF (sigla da unidade federativa), COD\_MUNICIPIO (código do município), LOCALIZACAO (urbana ou rural) E DEP\_ADMINISTRATIVA (instituição privada ou pública), consideradas chave primária composta do conjunto de dados. Deste modo, foi construído uma única estrutura de dados tabular (\textit{dataframe}) contendo as variáveis institucionais relevantes à manifestação da evasão na fase final da educação básica (BANAAG ET AL., 2024; SHIRASU \& ARRAES, 2018; SOUSA ET AL., 2025). 

Para modelagem estatística ulterior isolada também foi coletada uma amostra a nível escolar do Sistema de Avaliação da Educação Básica (SAEB), com variáveis sociais e acadêmicas no ano de 2017, também oriunda dos microdados do INEP.

A base de dados da PNAD foi obtida diretamente a partir do website do IBGE na seção PNAD Contínua e lida preliminarmente em R para interpretação dos dados de largura fixa a largura variável com o auxílio da biblioteca PNADcIBGE. Foram filtrados apenas os resultados pertinentes, a partir do de ano de 2016 até 2024, à analise de ocorrência de pessoas que não frequentam mais a escola, frequentaram escola alguma vez, cursaram como grau mais elevado o ensino médio regular ou 2º grau, e com idade até 18 anos. Com base nesses dados é possível rotular ocorrência de evasão ou conclusão do ensino médio como variável binária exclusiva dependente baseada na conclusão do curso.

Após o procedimento de filtragem, foi feita uma seleção das características, ou colunas, mais relevantes para explicar o fenômeno da evasão escolar. Características como sexo, cor ou raça, se já trabalhou ou estagiou por pelo menos 1 hora em alguma atividade remunerada em dinheiro, remunerada em mercadorias e bens, não remunerada ou atividade ocasional ("bico"), número de componentes do domicílio (exclusive as pessoas cuja condição no domicílio era pensionista, empregado doméstico ou parente do empregado doméstico), se recebe bolsa família ou outro auxílio governamental, rendimento domiciliar per capita
(habitual de todos os trabalhos e efetivo de outras fontes), situação do domicílio, tipo de área (do domicílio), horas trabalhadas por semana em atividade principal, secundária e ocasional, se houve procura de emprego pelo entrevistado nos últimos 30 dias, por quanto tempo procura emprego e o tipo de abastecimento de água do domicílio, as quais são algumas das características que mais se correlacionam qualitativamente ao abandono e à evasão escolar no ensino médio (FERREIRA; OLIVEIRA, 2020).

Foi gerado um arquivo CSV com os dados já filtrados e características selecionadas, e posteriormente este foi lido em Python para limpeza de dados faltantes, e verificação dos tipos de variáveis com o objetivo de utilizá-lo como base de dados complementar à análise de evasão. Foi também criada a coluna "evasao" codificada de forma binária para classificar o aluno como evasão (1) e conclusão (0).


\subsection{Pré-processamento dos Dados}
As taxas obtidas a partir da base do INEP, inicialmente no formato 0 a 100 foram transformadas para formato decimal (0 a 1) como:

\vspace{0.5cm}
\begin{equation}
\large T_k = \frac{t_k}{100}
\end{equation}
\vspace{0.5cm}

Em que $T_k$ representa o vetor modificado e $t_k$ representa o vetor (coluna) da taxa original

Baseado nas limitações de tipo do modelo e da biblioteca utilizada, variáveis categóricas foram separadas em novas colunas de acordo com o número de valores únicos de texto presentes na coluna única original, transformadas cada coluna nova em vetores binários e na sequência a remoção de uma ou mais colunas para evitar correlação linear entre as características artificialmente separadas.  

Variáveis ordinais numéricas foram tranformadas em variáveis categóricas textuais e, em sequência, o fluxo de transformação citado anteriormente para variáveis categóricas foi aplicado.

Células vazias ou nulas tiveram suas linhas removidas por completo da base de modelagem para integridade da análise estatística.

\subsection{Análise de Multicolinearidade}
Multicolinearidade pode ser compreendida como uma relação linear entre duas ou mais variáveis dependentes. Conforme Paul (2006), quando há importância na investigação dos impactos dos regressores na variável dependente, a multicolinearidade pode ser um problema, visto que p-valores podem se mostrar equivocadamente elevados e em alguns casos pode interferir na interpretação dos coeficientes. Uma das maneiras de analisar se há multicolinearidade no conjunto de dados é calcular o fator de inflação de variância (\textit{VIF}), baseado em $R^2$, que indica o quanto uma variável independente é explicada pelos demais regressores para cada umas das $i$ variáveis indepedentes (MILOCA \& CONEJO, 2008).

No presente trabalho será aplicado o \textit{VIF} para verificar a mutlicolinearidade entre as variáveis advindas das bases do INEP:

\vspace{0.5cm}
\begin{equation}
\large F_i = \frac{1}{1-R_i^2}
\end{equation}
\vspace{0.5cm}


$F_i$ se refere ao fator de inflação de variância do i-ésimo regressor; $R_i^2$ se refere ao $R^2$ da i-ésima variável independente.

Um \textit{VIF} maior que 10 indica que a multicolinearidade influencia fortemente o valor dos coeficientes do modelo, como proposto por Johnson e Wichern (1988; apud MILOCA \& CONEJO, 2008), e algumas medidas em relação ao conjunto de dados ou ao modelo devem ser tomadas com o objetivo de preservar a interpretação dos dados na modelagem estatística.

\subsection{Modelo Estatístico}
Este artigo visa aplicar métodos estatísticos com o intuito de investigar o impacto das variáveis socioeconômicas e institucionais na evasão escolar e em variáveis que corroboram tal fenômeno. Para tanto, conforme Green et al (2011), que salientam a regressão linear como ferramenta amplamente usada para examinar as relações estatísticas entre variáveis, nesta pesquisa será utilizado, primordialmente, o modelo de regressão linear múltipla, expresso por:

\vspace{0.5cm}
\begin{equation}
\large Y = \beta_0 + \beta_1X_1 + \beta_2X_2 + \beta_3X_3 + \cdots + \beta_iX_i + \epsilon 
\end{equation}
\vspace{0.5cm}

Em que $Y$ é a variável dependente a ser modelada, $X_i$ são as variáveis independentes, ou regressores, $\beta_i$ representa os coeficientes atrelados a cada variável independente e $\epsilon$ representa o erro aleatório. 

Com o modelo, e o auxílio da técnica dos mínimos quadrados para quantificar os coeficientes, é possível mensurar a significância das diferentes variáveis independentes, e o quanto sua variação impacta na variável dependente, de acordo com o contexto das amostras e das variáveis coletadas.



\subsection{Métricas de Desempenho}
Com o objetivo de extrair resultados acerca do desempenho do modelo deste trabalho, foram utilizadas as técnicas $R^2$ e a média dos quadrados dos resíduos (\textit{MSE}). Por meio destas é possível ter ciência do quanto o modelo explica a variação da variável dependente que varia de 0 a 1 (sendo 1 o ajuste perfeito dos dados pelo modelo) e a média da magnitude de erros do modelo, respectivamente:

\vspace{0.5cm}
\begin{equation}
\large R^2 = 1 - \frac{\sum_i(y_i-\hat{y}_i)^2}{\sum_i(y_i-\bar{y})^2}
\end{equation}
\vspace{0.5cm}

\vspace{0.5cm}
\begin{equation}
\large E_{qm} = \frac{1}{n} \sum_i(y_i-\hat{y_i})^2
\end{equation}
\vspace{0.5cm}

De forma que $y_i$ é o valor real, $\hat{y_i}$ é valor predito pelo modelo e $\bar{y}$ é a média dos valores da variável dependente.

\section{Resultados Parciais}

Com base nos dados coletados a partir dos dados abertos do INEP e na regressão linear múltipla resumida na tabela 1:

\vspace{0.5cm}
\begin{table}[htbp]
\centering
\caption{Resultados da regressão OLS para a variável dependente \texttt{TX\_EV\_TOTAL}}
\label{tab:regressao_tx_ev_sem_ic}
\begin{tabular}{lrrrr}
\hline
\textbf{Variável} & \textbf{Coef.} & \textbf{Erro Padrão} & \textbf{t} & \textbf{P$>$|t|} \\
\hline
const & -0.0029 & 0.003 & -0.898 & 0.369 \\
DEP\_ADMINISTRATIVA\_Pública & 0.0273 & 0.002 & 16.393 & 0.000 \\
Grupo 2 & 0.0572 & 0.009 & 6.338 & 0.000 \\
Grupo 3 & -0.0115 & 0.004 & -3.189 & 0.001 \\
Grupo 4 & -0.0168 & 0.006 & -2.989 & 0.003 \\
Grupo 5 & 0.0049 & 0.005 & 1.079 & 0.281 \\
IN\_BIBLIOTECA & 0.0016 & 0.002 & 0.694 & 0.488 \\
IN\_COZINHA & -0.0031 & 0.003 & -1.143 & 0.253 \\
IN\_INTERNET & 0.0178 & 0.003 & 6.740 & 0.000 \\
IN\_LABORATORIO\_CIENCIAS & -0.0052 & 0.004 & -1.253 & 0.210 \\
IN\_LABORATORIO\_INFORMATICA & 0.0011 & 0.003 & 0.413 & 0.680 \\
IN\_QUADRA\_ESPORTES & 0.0025 & 0.003 & 0.881 & 0.378 \\
NUM\_ALUNO\_TURMA\_TOTAL & 0.0000 & 0.000 & 0.642 & 0.521 \\
TX\_DI\_TOTAL & 0.1853 & 0.004 & 41.654 & 0.000 \\
TX\_IED\_N5 & 0.0070 & 0.004 & 1.767 & 0.077 \\
TX\_IED\_N6 & -0.0112 & 0.006 & -1.939 & 0.053 \\
TX\_REP\_TOTAL & 0.0528 & 0.009 & 5.622 & 0.000 \\
\hline
\multicolumn{5}{l}{\footnotesize $R^2 = 0.514$, Estatística-F = 455.7, Prob(F) = 0.00, $E_{qm}=0.0026$} \\
\multicolumn{5}{l}{\footnotesize Observações = 6915} \\
\end{tabular}
\end{table}
\vspace{0.5cm}



Os resultados mostram que, ao manter as outras variáveis constantes, quando TX\_DI\_TOTAL, a taxa de defasagem idade-série, aumenta em 1 ponto percentual, a taxa de evasão total tende a aumentar em 0,1853 pontos percentuais. O coeficiente é estatisticamente significativo com $p<0.001$ e se mostra como o coeficiente mais alto que se relaciona com a variável dependente TX\_EV\_TOTAL.

De maneira semelhante, TX\_REP\_TOTAL, que demonstra a taxa de repetência total do município, também apresenta alto coeficiente em comparação com as outras variáveis, sendo $0,0528$, e estatisticamente significativo com $p<0,0001$.

Os resultados também sugerem que, com $N=6915$ observações, as variáveis de infraestrutura coletadas não se mostram, em sua maioria, como estatisticamente significadas, e aquela que apresentam p-valor menor que 5\% (IN\_INTERNET) exibe coeficiente $\beta_i<0,0178$, o que caracteriza uma ínfima variação de pontos percentuais na taxa de evasão total. Taxas do índice de esforço do docente (TX\_IED\_N5 e TX\_IED\_N5) de forma análoga não se mostram estatisticamente significativos, com $p>0,05$.

Em resumo, as variáveis Grupo 2, TX\_DI\_TOTAL, TX\_REP\_TOTAL, são evidenciadas como as mais relevantes no presente estudo. 

Baseado, em particular, na taxa de repetência, decidiu-se utilizar a base do Sistema de Avaliação do Educação básica (SAEB) do INEP, para descobrir quais variáveis mais impactam na taxa de repetência e, de forma implícita, na taxa de defasagem idade-série.

Com base na tabela 2, que resume os resultados da regressão linear múltipla da base de dados do SAEB:

\vspace{0.5cm}
\begin{table}[htbp]
\centering
\caption{Resultados da regressão OLS para a variável dependente \texttt{TX\_RESP\_Q041}}
\label{tab:regressao_tx_resp_q041}
\begin{tabular}{lrrrrr}
\hline
\textbf{Variável} & \textbf{Coef.} & \textbf{Erro Padrão} & \textbf{t} & \textbf{P$>$|t|} & \textbf{[0.025, 0.975]} \\
\hline
const & 0.2779 & 0.011 & 25.240 & 0.000 & 0.256 -- 0.299 \\
PROFICIENCIA\_MT & -0.1585 & 0.003 & -62.430 & 0.000 & -0.164 -- -0.154 \\
TX\_RESP\_Q001 & 0.0852 & 0.010 & 8.284 & 0.000 & 0.065 -- 0.105 \\
TX\_RESP\_Q002\_Indígena & 0.1252 & 0.018 & 6.794 & 0.000 & 0.089 -- 0.161 \\
TX\_RESP\_Q002\_Parda & 0.0514 & 0.007 & 7.256 & 0.000 & 0.037 -- 0.065 \\
TX\_RESP\_Q002\_Preta & 0.1439 & 0.012 & 12.093 & 0.000 & 0.121 -- 0.167 \\
TX\_RESP\_Q038 & 0.0639 & 0.007 & 8.602 & 0.000 & 0.049 -- 0.078 \\
TX\_RESP\_Q044 & -0.1061 & 0.010 & -10.970 & 0.000 & -0.125 -- -0.087 \\
TX\_RESP\_Q052 & 0.0914 & 0.009 & 10.613 & 0.000 & 0.074 -- 0.108 \\
\hline
\multicolumn{6}{l}{\footnotesize $R^2 = 0.268$, Estatística-F = 861.3, Prob(F) = 0.00} \\
\multicolumn{6}{l}{\footnotesize Observações = 18846} \\
\end{tabular}
\end{table}
\vspace{0.5cm}

Pode-se observar que das variáveis coletadas todas são significativas com $p<0,001$, e se destacam as variáveis PROFICIENCIA\_MT ($\beta_i=-0,1585$), que mede a proficiência de matemática média da instituição; TX\_RESP\_Q002\_Preta ($\beta_i=0,1439$), que refere-se à taxa de alunos autodeclarados pretos na instituição de ensino; TX\_RESP\_Q002\_Indígena ($\beta_i=0,1252$), que faz juz à taxa de alunos autodeclarados indígenas. Tais variáveis possuem os maiores coeficientes do conjunto de regressores. 

De acordo com os resultados, o aumento de 1 ponto na proficiência em matemática faz com que a taxa de repetência tenda a diminuir em 0,1585 pontos percentuais.

Por outro lado, o aumento de 1 ponto percentual no grupo de alunos autodeclarados pretos nas instituições tende a elevar a taxa de repetência em 0,1439. E o mesmo pode ser observado para o grupo de alunos indígenas, cujo aumento de 1 ponto percentual implica em $+0.1252$ pontos percentuais na taxa de repetência.

A variável TX\_RESP\_Q041, que faz referência à taxa de alunos que gostam de estudar língua portuguesa, também mostra que com o aumento de 1 ponto percentual, a taxa de repetência tende a $-0,1061$ pontos percentuais.

A correlação entre evasão, repetência e defasagem idade-série é corroborada pela cenário do ensino médio brasileiro. De acordo com dados da PNAD, pesquisa nacional realizada pelo IBGE, a taxa de evasão entre pretos em idade escolar adequada ao ensino secundário é de 22,87\%, entre pardos 22,26\%, e entre indígenas equivalente a 25,64\%, com base na lógica de cálculo de evasão deste trabalho. Brancos apresentam o menor número com cerca de 16,16\%. Na repetência, o grupo de pretos e pardos é cerca de 2 a 4 pontos percentuais acima do percentual de brancos que se mantêm na mesma série em todo o Brasil. Ainda segundo a plataforma inepdata, a taxa de jovens pretos e pardos que estão defasados no quesito idade-série, chega a ser de 10 pontos percentuais em comparação com jovens brancos.

A evasão e dificuldade escolar de tais grupos fragilizados também são refletidos em sua situação socioeconômica e o inverso também é verdadeiro. Jovens em idade escolar pretos apresentam renda média mensal de 790 reais, jovens pardos 780 reais, jovens indígenas 727 reais, e jovens brancos 1230 reais em média, segundo dados da PNAD e no contexto desta pesquisa.






%====================================================================


%See the guidelines for metadata and references:
%https://sol.sbc.org.br/journals/index.php/rbie/libraryFiles/downloadPublic/71
%====================================================================

\pagebreak
\nocite{*}
\printbibliography


\end{document}