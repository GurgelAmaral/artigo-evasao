%
% Template for RBIE papers in LaTeX
%

% The above language combination is for this template document only.
% You should use one of the following:
\documentclass[english, spanish, brazilian]{RBIEarticle} % for papers in portuguese
%\documentclass[brazilian, spanish, english]{RBIEarticle} % for papers in english
%\documentclass[brazilian, english, spanish]{RBIEarticle} % for papers in spanish

% Papers in Portuguese or Spanish may require the following lines:
\usepackage[utf8]{inputenc} % chooses UTF-8 as the main character set
\usepackage[T1]{fontenc} % for correct syllable separation in accented words
% Pacotes para citações/referências ABNT
%usepackage[alf]{abntex2cite} % citações autor-data
%\usepackage[num]{abntex2cite} % citações numéricas
\usepackage{amsmath}

% The next two statements are needed for the example table in this document
% (i.e. you don't necessarily need them in your own paper)
\usepackage{colortbl}
\definecolor{gray}{gray}{.8}

% Citations and references (Biblatex)
% Citations and references (Biblatex)
\usepackage[style=abnt]{biblatex}
\usepackage{csquotes}
\addbibresource{references.bib}

% Here goes the paper main title
\title{Aprendizado de Máquina Aplicado à Predição da Evasão no Ensino Médio Brasileiro: Uma Abordagem Baseada em Variáveis Socioeconômicas}

% If the manuscript is written in English, then this element must be removed.
\titleinenglish{Machine Learning Applied to Predicting Dropout Rates in Brazilian High Schools: An Approach Based on Socioeconomic Variables}

% If the manuscript is written in English, then this element must be removed.
\titleinspanish{Aprendizaje automático aplicado a la predicción de la deserción escolar en la educación secundaria brasileña: un enfoque basado en variables socioeconómicas}

% Here goes the paper author information (repeat for two or more authors)
\author{%
	\parbox{3.8cm}{%
		Bruno Alexandre Dias da Silva\\
		Universidade de São Paulo\\
		ORCID: \href{https://orcid.org/0000-0000-0000-0000}{0000-0000-0000-0000}\\
		Brunoalexdias20@usp.br
	}
        \hspace{0.3cm}
	\parbox{3.8cm}{%
		Lucas Gurgel do Amaral\\
		Universidade de São Paulo\\
		ORCID: \href{https://orcid.org/0000-0000-0000-0000}{0000-0000-0000-0000}\\
		lucasgurgel@usp.br
	}
        \hspace{0.3cm}
        \parbox{3.8cm}{%
		Rafael de França\\
		Universidade de São Paulo\\
		ORCID: \href{https://orcid.org/0000-0000-0000-0000}{0000-0000-0000-0000}\\
		rafaeldefranca@usp.br
	}
        \hspace{0.3cm}
	\parbox{3.9cm}{\raggedright%
		Richard Pereira do Nascimento\\
		Universidade de São Paulo\\
		ORCID: \href{https://orcid.org/0000-0000-0000-0000}{0000-0000-0000-0000}\\
		rcdwoods@usp.br
	}
}

\Submission{dd/Mmm/yyyy}
\First_round_notif{dd/Mmm/yyyy}
\New_version{dd/Mmm/yyyy}
\Second_round_notif{dd/Mmm/yyyy}
\Camera_ready{dd/Mmm/yyyy}
\Edition_review{dd/Mmm/yyyy}
\Available_online{dd/Mmm/yyyy}
\Published{dd/Mmm/yyyy}

% Here goes the page heading information
\heading{Gurgel et al.
}{RBIE v.VV – 2025}

% And finally here goes the citation information
\citeas{Last name, Initials., \ldots \& Last name, Initials.  (Year). Article title in the original language. Revista Brasileira de Informática na Educação, vol, pp-pp. https://doi.org/10.5753/rbie.yyyy.id}

\citeas{
SILVA, B. A. D.; GURGEL, L.; FRANÇA, R. D.; NASCIMENTO, R. P. do. Aprendizado de Máquina Aplicado à Predição de Evasão no Ensino Médio em São Paulo. Revista Brasileira de Informática na Educação, vol , pp-pp, 2025.
}

%====================================================================
%\hyphenpenalty=10000
%\setcounter{page}{01}

\begin{document}
\maketitle

% If the manuscript is written in English, then this element must be removed.
\begin{otherlanguage}{brazilian}
\begin{abstract}
A evasão escolar no ensino brasileiro vem se mostrando como um grande desafio na qualificação e formação educacional do público infanto-juvenil, principalmente nas camadas menos favorecidas da sociedade brasileira. Jovens que não concluem o ensino médio não conseguem especializar-se em cursos superiores e, portanto, submetem-se a empregos com mão de obra barata, perdurando um ciclo de pobreza e baixos indicadores socioeconômicos. Para isso, este trabalho tem como objetivo avaliar e comparar a relevância de variáveis socioeconômicas e intitucionais, a fim de desenvolver modelos probabilísticos de aprendizado de máquina com o intuito de prever a probabilidade de um aluno evadir o ensino médio da rede de ensino brasileira, baseado em técnicas como Floresta Aleatória, Árvores de Decisão e Redes Neurais. Por meio da base de microdados e dados abertos do Instituto Nacional de Estudos e Pesquisas Educacionais Anísio Teixeira (INEP), e da base de dados da Pesquisa Nacional por Amostra de Domicílios (PNAD) provenientes do Instituto Brasileiro de Geografia e Estatística (IBGE), serão extraídas variáveis que, de acordo com a literatura de estudos acerca de evasão, mais expliquem o abandono escolar contínuo. Os resultados deste artigo podem auxiliar docentes e escolas a rastrearem e prestarem mais apoio àqueles alunos que apresentam alta chance de evasão por motivos de cunho social, econômico, acadêmico e/ou institucional.
\keywords\ Evasão escolar; Aprendizado de Máquina; PNAD; Variáveis Socioeconômicas; Variáveis Institucionais; Inep; Ensino médio.
\end{abstract}
\end{otherlanguage}

\begin{otherlanguage}{english}
\begin{abstract}
School dropout rates in Brazil have proven to be a major challenge in the educational qualification and training of children and young people, especially in the less privileged segments of brazilian society. Young people who do not complete high school are unable to specialize in higher education courses and, therefore, take on low-wage jobs, perpetuating a cycle of poverty and low socioeconomic indicators. Therefore, this study aims to evaluate and compare the significance of socioeconomic and institutional variables, in order to develop probabilistic machine learning models to predict the probability of a student dropping out of high school in the Brazil school system based on techniques such as Random Forest, Decision Trees, and Neural Networks. Using data from the Anísio Teixeira National Institute for Educational Studies and Research (INEP), and from the Brazilian Institute of Geography and Statistics (IBGE) National Household Sample Survey (PNAD), variables will be extracted as they, according to the literature on dropout, best explain continuous school abandonment. The results of this article can help teachers and schools track and provide more support to those students who are at high risk of withdrawal due to social, economic, academic and/or institutional reasons.
\keywords\ School dropout; Machine learning; PNAD; Socioeconomic variables; Institutional variables; Inep ;Secondary education.
\end{abstract}
\end{otherlanguage}

% If the manuscript is written in English, then this element must be removed.
\begin{otherlanguage}{spanish}
\begin{abstract}
La deserción escolar en la educación brasileña se ha convertido en un gran desafío para la calificación y la formación educativa de los niños y jóvenes, especialmente en los sectores menos favorecidos de la sociedad brasileña. Los jóvenes que no completan la educación secundaria no pueden especializarse en cursos superiores y, por lo tanto, se ven obligados a aceptar empleos con mano de obra barata, perpetuando un ciclo de pobreza y bajos indicadores socioeconómicos. Por ello, este trabajo tiene como objetivo evaluar y comparar la relevancia de variables socioeconómicas e institucionales, con el fin de desarrollar modelos probabilísticos de aprendizaje automático con el fin de predecir la probabilidad de que un estudiante abandone la educación secundaria en la red educativa brasileña, basándose en técnicas como bosques aleatorios, árboles de decisión y redes neuronales. A través de la base de microdatos y datos abiertos del Instituto Nacional de Estudios e Investigaciones Educativas Anísio Teixeira (INEP), y de la base de datos de la Encuesta Nacional por Muestra de Hogares (PNAD) del Instituto Brasileño de Geografía y Estadística (IBGE), se extraerán variables que, según la literatura de estudios sobre el abandono escolar, explican mejor el abandono escolar continuo. Los resultados de este artículo pueden ayudar a los docentes y a las escuelas a rastrear y brindar más apoyo a aquellos alumnos que presentan un alto riesgo de deserción por motivos de índole social, económica, académica y/o institucional.
\keywords\ Absentismo escolar; Aprendizaje automático; PNAD; Variables socioeconómicas; Variables institucionales; INEP; Educación secundaria.
\end{abstract}
\end{otherlanguage}

\pagebreak

%====================================================================

\section{Introdução}

A evasão escolar é definida como a ausência de retorno ao sistema de ensino formal do aluno em idade escolar após abandono ou reprovação. Deste modo, enquanto o abandono faz referência à situação do aluno que deixa de frequentar as aulas durante o ano letivo, a evasão refere-se ao aluno que, por qualquer motivo, não regressou à rede de ensino com o reinício do ano letivo, assim como descrito pelo manual guia do Sistema de Alerta Preventivo de Evasão e Abandono Escolar (SAP).

Elucidada sua definição, a evasão se mostra um grande desafio na esfera do ensino brasileiro, em especial ao ensino público. Segundo os indicadores de taxa de transição e fluxo provenientes do Censo Escolar, em nível nacional e na etapa do ensino médio, a taxa de evasão apresenta valores preocupantes: enquanto a rede privada registra índices de aproximadamente 2\% a 3\% na série histórica, a rede pública chega a patamares entre 7\% e 12\% do contingente total de alunos (INEP, 2022). No Brasil como um todo, de acordo com dados do IBGE (2021), cerca de 5 milhões de jovens entre 14 e 29 anos estavam fora da escola, sendo que uma parcela significativa não havia concluído o ensino médio.

A literatura especializada evidencia que o fenômeno da evasão é multifatorial. Silva (2013) ressalta que o nível socioeconômico das famílias é um dos principais fatores que impele estudantes a não retomar a vida acadêmica. Um baixo nível socioeconômico, que reflete também a pobreza, induz indivíduos a abandonar os estudos à procura de trabalho para complementar a renda familiar mensal. A dificuldade em conciliar estudo e trabalho, a necessidade de contribuir economicamente para o domicílio e a falta de suporte institucional figuram como causas recorrentes. Além disso, a ausência de políticas públicas eficazes de acompanhamento e prevenção agrava o problema, pois não há um sistema consolidado de rastreamento que permita identificar estudantes em risco de evasão.

Do ponto de vista das consequências sociais, a evasão escolar impacta diretamente a qualificação da mão de obra e perpetua ciclos de pobreza. Jovens que não concluem o ensino médio enfrentam maiores barreiras de acesso ao ensino superior e ficam restritos a ocupações de baixa remuneração e instabilidade, reduzindo suas chances de mobilidade social. Segundo o IBGE (2021), a renda média de um trabalhador com ensino médio completo pode ser até 40\% superior à de um trabalhador que abandonou os estudos antes de concluir essa etapa. Dessa forma, o fenômeno não afeta apenas o indivíduo, mas também repercute nos indicadores socioeconômicos do país, ampliando desigualdades estruturais.

Comparações internacionais reforçam a gravidade do cenário brasileiro. Dados da UNESCO (2020) mostram que o Brasil apresenta taxas de abandono escolar superiores à média de países da América Latina e muito distantes das registradas em nações da OCDE, onde os índices de evasão no ensino médio são inferiores a 5\%. Essa discrepância aponta para a necessidade de maior investimento em políticas educacionais de permanência, associadas ao monitoramento de fatores de risco.

Diante desse panorama, estudos que integrem variáveis socioeconômicas e técnicas computacionais de análise se mostram fundamentais. A Pesquisa Nacional por Amostra de Domicílios (PNAD), realizada pelo IBGE, constitui uma das principais fontes de dados para compreender o perfil dos estudantes e identificar as condições de vulnerabilidade associadas ao abandono escolar. Ao possibilitar a análise de fatores como renda domiciliar, escolaridade dos responsáveis, ocupação e acesso a programas sociais, a PNAD oferece subsídios robustos para a construção de modelos preditivos.

Nesse sentido, este artigo investiga a problemática da evasão escolar no Brasil a partir da PNAD, com a finalidade de analisar e desenvolver modelos probabilísticos baseados em aprendizado de máquina. O objetivo é avaliar a probabilidade de evasão escolar de alunos com base em suas características socioeconômicas, a fim de proporcionar um método de predição capaz de auxiliar gestores educacionais na implementação de estratégias preventivas e compreender de que forma o perfil do estudante impacta seu desenvolvimento acadêmico e social.



\section{Fundamentos Teóricos}

O presente capítulo tem como objetivo apresentar a base conceitual que sustenta esta pesquisa, fornecendo o embasamento necessário para compreender as abordagens utilizadas na predição da evasão escolar no Ensino Médio. A fundamentação teórica é fundamental para situar o estudo no contexto das pesquisas já existentes, permitindo identificar conceitos-chave, modelos consolidados e contribuições de autores que exploraram temas relacionados à evasão escolar e ao uso de técnicas de aprendizado de máquina em contextos educacionais.

\subsection{Evasão Escolar e Fatores Socioeconômicos}
A evasão escolar é um fenômeno amplamente estudado nas áreas da Educação e das Ciências Sociais, podendo ser associado a fatores de ordem social, econômica e cultural. Segundo Araque, Roldán e Salguero (2009), as principais razões que influenciam o abandono da escola estão relacionadas a variáveis socioeconômicas, familiares, institucionais e psicológicas. Nesse sentido, compreender esse fenômeno requer considerar tanto o contexto social do aluno quanto as condições oferecidas pelo sistema educacional.

\subsection{Mineração de Dados Educacionais e Aprendizado de Máquina}
No campo da análise de dados, o termo Mineração de Dados Educacionais (\textit{Educational Data Mining — EDM}) refere-se ao uso de métodos computacionais para explorar grandes volumes de dados gerados em ambientes educacionais, buscando padrões relevantes (BAKER; ISOTANI; CARVALHO, 2011). O objetivo disso é apoiar a tomada de decisão em políticas públicas e institucionais, fornecendo evidências que permitam compreender e reduzir problemas como a própria evasão escolar.

Associado à EDM, o Aprendizado de Máquina (\textit{Machine Learning}) é um subcampo da Inteligência Artificial que possibilita a criação de modelos preditivos a partir de dados. Esses modelos são capazes de identificar relações complexas entre variáveis e gerar previsões com base em informações históricas (MITCHELL, 1997). No contexto da evasão escolar, técnicas de aprendizado supervisionado permitem estimar a probabilidade de um estudante abandonar ou concluir seus estudos, dadas suas características socioeconômicas, acadêmicas e pessoais (TEODORO; KAPPEL, 2020).

\subsection{Principais Técnicas Utilizadas}
Entre as abordagens mais comuns em problemas de classificação binária, destacam-se a Regressão Logística, as Redes Neurais e os métodos de Floresta Aleatória de Classificação. A Regressão Logística (RADÜNZ, 1992) permite modelar a relação entre variáveis independentes e uma variável dependente categórica, atribuindo probabilidades ao evento de interesse. Já as Redes Neurais, inspiradas no funcionamento do cérebro humano, são capazes de identificar padrões não-lineares e de alta complexidade (GARDNER; DORLING, 1998). Por fim, o método de Florestas Aleatórias, proposto por Breiman (2001), combina múltiplas árvores de decisão para melhorar a acurácia preditiva e reduzir problemas de sobreajuste (\textit{overfitting}).

\subsection{Técnicas Computacionais Complementares}
Embora este trabalho foque na aplicação de Regressão Logística, Redes Neurais e Florestas Aleatórias de Classificação, outras técnicas computacionais também têm sido amplamente empregadas em estudos de predição e análise de dados educacionais. Entre elas, destacam-se:

\textbf{Máquinas de Vetores de Suporte (SVM)}: modelo supervisionado que busca encontrar um hiperplano ótimo de separação entre classes. As SVMs são reconhecidas por sua eficiência em problemas de classificação binária e por sua capacidade de lidar com dados de alta dimensionalidade, sendo frequentemente utilizadas em contextos educacionais para a identificação de padrões de desempenho acadêmico (CORTES; VAPNIK, 1995).

\textbf{Naive Bayes}: técnica baseada em probabilidade que utiliza o Teorema de Bayes para estimar a classe de uma observação com base na distribuição condicional das variáveis independentes. Apesar de sua simplicidade e da suposição de independência entre atributos, o Naive Bayes apresenta bons resultados em problemas de classificação de texto e dados categóricos, podendo ser aplicado à análise de questionários e registros escolares (ZHANG, 2004).

\textbf{K-Nearest Neighbors (KNN)}: algoritmo que classifica uma observação com base na proximidade de seus vizinhos mais próximos em um espaço multidimensional. O KNN é intuitivo e eficaz em situações em que não se pressupõe uma relação linear entre as variáveis, sendo útil para identificar perfis de estudantes com risco de evasão a partir de características socioeconômicas semelhantes (COVER; HART, 1967).

Além dessas técnicas, também se destacam os métodos de \textit{clustering} como o K-Means, utilizados em contextos de aprendizado não supervisionado. Esses métodos permitem segmentar grupos de estudantes com características semelhantes, favorecendo análises exploratórias sobre padrões de evasão sem a necessidade de rótulos previamente definidos.

\subsection{A Pesquisa Nacional por Amostra de Domicílios (PNAD)}
A Pesquisa Nacional por Amostra de Domicílios (PNAD), realizada pelo Instituto Brasileiro de Geografia e Estatística (IBGE), constitui uma das principais fontes de dados socioeconômicos no Brasil. Seu objetivo é coletar informações abrangentes sobre características demográficas, educacionais, ocupacionais e de rendimento da população brasileira, por meio de entrevistas domiciliares aplicadas em amostras representativas em nível nacional, regional e estadual (IBGE, 2022).

A PNAD Contínua, em vigor desde 2012, aprimorou o levantamento ao adotar coleta trimestral, permitindo análises mais atualizadas e consistentes acerca da dinâmica social e econômica do país. Entre suas variáveis, destacam-se renda familiar per capita, inserção no mercado de trabalho, características do domicílio, composição familiar, escolaridade e acesso a programas sociais. Tais informações são de grande relevância para estudos sobre evasão escolar, pois possibilitam identificar relações entre vulnerabilidade socioeconômica e permanência na escola.

\subsection{Importância da PNAD no Contexto Acadêmico}
A PNAD desempenha papel fundamental como fonte de dados em pesquisas acadêmicas brasileiras, especialmente nas áreas de Ciências Sociais, Economia e Educação. Por sua abrangência e representatividade estatística, a pesquisa possibilita análises robustas sobre condições de vida, desigualdade e mobilidade social. Estudos desenvolvidos por instituições de ensino e pesquisa frequentemente utilizam a PNAD como base empírica para compreender fenômenos relacionados ao trabalho, à renda e à escolaridade da população.

No campo educacional, a PNAD se destaca por permitir a investigação de variáveis socioeconômicas associadas à permanência ou evasão escolar. A inclusão de indicadores como nível de escolaridade dos responsáveis, rendimento domiciliar per capita, participação no mercado de trabalho e acesso a programas sociais fornece subsídios para estudos que buscam identificar fatores de risco e vulnerabilidade entre os estudantes. Adicionalmente, a consolidação histórica da PNAD garante comparabilidade ao longo do tempo, permitindo a construção de séries históricas sobre aspectos socioeconômicos.

\subsection{Síntese}
Dessa forma, os fundamentos apresentados evidenciam a relevância de associar teorias sobre evasão escolar com métodos de mineração de dados e aprendizado de máquina, bem como o uso de bases de dados socioeconômicas amplas como a PNAD. Essa integração permite analisar grandes quantidades de dados e construir modelos preditivos capazes de apoiar políticas públicas e institucionais na área da Educação, contribuindo para a redução da evasão escolar e para a promoção da equidade educacional.

\section{Trabalhos Relacionados}
Diversos estudos buscaram modelar de forma probabilística ou com técnicas de aprendizado de
máquina o fenômeno da evasão escolar no Brasil. Como exemplo, destacam dois trabalhos acerca
da evasão no âmbito acadêmico.

Em Mello et al. (2023) foram realizadas análises exploratórias acerca das características
consideradas (variáveis independentes) como coeficiente de rendimento, tipo de escola de origem,
percentual de frequência, nível de ensino, cor ou raça, modalidade de curso, sexo, renda familiar
bruta, situação acadêmica, dentre outros aspectos, de alunos do Instituto Federal do Pernambuco
(IFPE). Com a análise constatou-se que, como alguns exemplos, alguns cursos e períodos apresentam
evasão maior que outros; o percentual de evasão do sexo masculino era maior em comparação ao
feminino. Contudo, não houve análise acerca da renda per capita dos indivíduos, que é reiterada
por autores que exploraram o tema de forma qualitativa, como Silva (2013) e Ferreira \& Oliveira
(2020). Os autores utilizaram como técnica de aprendizado de máquina para a previsão de evasão
XGBoost, Florestas Aleatórias e Árvores de decisão, para predizer a situação de matrícula do
estudante conforme a base de dados disponível, e obtiveram, respectivamente, 82\%, 83\% e 80\%
de acurácia. Esses resultados sugerem que tais modelos apresentam ótimo desempenho preditivo
para estimar as probabilidades de evasão escolar, em concordância com o objetivo deste trabalho.

Por outro lado, o trabalho de Teodoro e Kappel (2020) também segue a mesma lógica do
trabalho supracitado, com a ressalva de que tem ênfase em atributos diferentes e apresenta uma
análise exploratória mais robusta acerca das características correlacionadas à evasão. Teodoro e
Kappel optaram por desenvolver modelos preditivos baseados em Naive Bayes, KNN, Árvores de
Decisão, Florestas Aleatórias de Classificação e Redes Neurais, com acurácias de 60\%, 75\%,
77\%, 79\% e 78\%, respectivamente, no geral. A apuração mostra que, em termos de acurácia,
Florestas Aleatórias e Redes Neurais são as técnicas mais adequadas para estimação de evasão.
Porém, assim como o estudo anterior, tem como principal objeto de estudo a evasão no ensino
superior.

Apesar de trabalhos aplicados ao ensino médio/técnico (por exemplo, Barbosa et al., 2023),
observa-se que a literatura nacional permanece concentrada em bases institucionais e recortes
locais. Em âmbito brasileiro, ainda há escassez de investigações que integrem variáveis
socioeconômicas de abrangência nacional — como renda per capita, trabalho precoce e insegurança
alimentar — a modelos preditivos para o ensino médio nas redes pública e privada. Este artigo
procura preencher essa lacuna ao combinar PNAD e aprendizado de máquina para estimar
probabilidades de risco de evasão.

Além das características individuais e domiciliares capturadas pela PNAD, a literatura indica que
fatores estruturais ao nível da escola também se associam ao risco de evasão no ensino médio.
Evidências reportam relações entre escolas de maior porte, turmas numerosas e razões aluno-
professor elevadas com maior probabilidade de abandono, sugerindo que variáveis de contexto
educacional podem complementar os atributos socioeconômicos no modelo preditivo. Em um
desenho nacional, tais elementos podem ser operacionalizados por indicadores derivados de bases
administrativas educacionais, agregados por unidade da federação ou município e vinculados aos
registros individuais via chaves geográficas.

Outro conjunto de determinantes envolve práticas e fluxos escolares. A literatura brasileira
relaciona a repetência/atraso escolar, a disponibilidade de vagas/turnos e a qualificação docente a
maiores taxas de abandono, o que respalda a inclusão de medidas como a taxa de reprovação no
ensino médio e a proporção de docentes com formação adequada como variáveis de contexto. Esses
indicadores, agregados por recortes territoriais, permitem capturar diferenças institucionais
relevantes sem exigir a identificação da escola no nível do indivíduo, preservando a modelagem em
escala nacional.

Condições de infraestrutura educacional — como conectividade, laboratórios e manutenção
predial — também são associadas a níveis de engajamento estudantil e, por consequência, à evasão.
Uma estratégia pragmática consiste em sintetizar um índice de infraestrutura (via combinação
simples de itens ou redução de dimensionalidade) por município ou unidade da federação e
integrá-lo como variável de contexto ao lado dos determinantes socioeconômicos da PNAD. Esse
acoplamento amplia a cobertura explicativa do modelo ao articular condições de vida e ambiente
escolar.

Em termos de interpretação e uso, trabalhos aplicados destacam o papel de técnicas de *ensemble*
e análises de *feature importance* para orientar intervenções pedagógicas e de gestão, ao mesmo
tempo em que estudos qualitativos em ensino médio ajudam a elucidar por que variáveis como renda
per capita, trabalho precoce e insegurança alimentar emergem como relevantes. Essa combinação de
previsibilidade e interpretabilidade é coerente com o objetivo de apoiar decisões em escala
nacional, mantendo atenção às diferenças regionais e às especificidades das redes.

Em linha com a ênfase em determinantes contextuais, evidências regionais analisando o ensino
médio público do Ceará, com modelos logísticos multiníveis, exploram simultaneamente
características dos estudantes e das escolas. Os resultados destacam a contribuição de fatores como
desinteresse declarado, histórico de repetência e defasagem idade-série para o aumento do risco de
abandono, enquanto a participação em programas de transferência de renda apresenta associação
negativa com a evasão. Esse tipo de achado complementa o enfoque nacional ao indicar que
variáveis de fluxo escolar e proteção social ajudam a explicar variações do risco para além das
condições domiciliares.

Já em aplicações baseadas em dados administrativos, há estudo recente com a rede estadual
paulista que estrutura um sistema de alerta precoce com classificadores de aprendizado de máquina
(por exemplo, florestas, *gradient boosting* e regressão logística) a partir de histórico escolar e
registros de frequência. Além de reforçar o bom desempenho de métodos de *ensemble*, o trabalho
relata ganhos práticos ao combinar importância de atributos com regras de priorização de
atendimento, o que dialoga com a proposta deste artigo de aliar previsibilidade e utilidade para a
gestão em escala nacional.

Questões metodológicas recorrentes. A literatura destaca desafios que também norteiam nossa
abordagem: (i) desbalanceamento de classes — a evasão costuma ser minoria; (ii) métricas
alinhadas ao objetivo de intervenção — acurácia isolada pode mascarar falsos negativos; (iii)
explicabilidade — necessária para a adoção por equipes pedagógicas; e (iv) transferência temporal
— modelos treinados em um período podem degradar com mudanças de coorte/currículo. Em linha
com essas recomendações, adotamos métricas como F1-score, AUC-ROC e, quando pertinente,
AUC-PR, priorizando sensibilidade/*recall* na classe de risco; e técnicas de reamostragem
(subamostragem, conforme nossa Metodologia) quando cabível. A análise de *feature importance*
é tratada como insumo interpretável para orientar ações, em consonância com práticas relatadas em
estudos aplicados (Barbosa et al., 2023; Teodoro \& Kappel, 2020).

Implicações para variáveis e desenho de intervenções. Estudos qualitativos ajudam a explicar por
que determinadas variáveis aparecem entre as mais relevantes nos modelos. Fatores como renda,
trabalho e alimentação são recorrentemente associados à decisão de abandonar a escola no ensino
médio brasileiro. Ao incorporar essas dimensões — operacionalizadas aqui via PNAD — produz-se
um modelo que não apenas prediz, mas também dialoga com mecanismos plausíveis de evasão
descritos na literatura (Silva, 2013; Ferreira \& Oliveira, 2020), o que favorece o desenho de
intervenções focalizadas (auxílio financeiro, merenda/reforço de programas de alimentação,
mediação de estágio/trabalho protegido, acompanhamento social).

Considerações de adoção e equidade. A transferência dos resultados para a prática escolar exige
atenção à calibração de probabilidades (para que o limiar de alerta reflita custos/benefícios de
intervenção) e à equidade (evitar que o modelo reforce desigualdades ao superidentificar grupos
específicos). O uso de métricas sensíveis à classe minoritária (Strauss; Villas Bôas Júnior;
Ferreira, 2022) e a verificação de viés por subgrupos (sexo, raça/cor) ajudam a balizar decisões
responsáveis. Tais cuidados metodológicos aumentam a aceitabilidade do sistema por gestores e
docentes e estão alinhados às melhores práticas de *Learning Analytics* aplicadas à educação
básica.

Nota de escopo. Embora a revisão sintetize achados nacionais e internacionais, o objetivo empírico
deste trabalho permanece delimitado ao ensino médio brasileiro, abrangendo as redes pública e
privada, em escala nacional. As referências mais amplas servem para contextualizar o estado da
arte e não implicam generalização automática para períodos futuros ou outros contextos;
discutimos validade externa e possíveis diferenças regionais nas seções de Resultados e
Conclusões.

Em complemento, estudos qualitativos voltados especificamente ao ensino médio contribuem para
fundamentar a seleção de variáveis e o enquadramento do problema. Batista, Souza e Oliveira
(2009) apresentam um estudo de caso que discute fatores estruturais, familiares e socioeconômicos
associados ao abandono na etapa final da educação básica. Embora não empregue técnicas de
aprendizado de máquina, o trabalho oferece evidências sobre a relevância de indicadores como
renda per capita, trabalho precoce e insegurança alimentar, que dialogam diretamente com as
variáveis extraídas da PNAD neste estudo. Assim, a literatura qualitativa auxilia a interpretabilidade
dos modelos, ao delinear mecanismos plausíveis pelos quais o contexto social do discente
influencia a probabilidade de evasão.

Do ponto de vista metodológico, a literatura também enfatiza a necessidade de métricas adequadas
para avaliar modelos preditivos em cenários com classes desbalanceadas, como a evasão (menos
frequente do que a permanência). Strauss, Villas Bôas Júnior e Ferreira (2022) discutem as
limitações da acurácia como métrica isolada e recomendam o uso de F1-score, AUC-ROC e, quando
pertinente, AUC-PR para capturar de forma mais fiel o desempenho sobre a classe minoritária.
Essas diretrizes sustentam as escolhas de avaliação adotadas neste trabalho e reforçam a importância
de reportar métricas sensíveis a falsos negativos, dada a natureza de risco e intervenção associada ao
problema.



\section{Metodologia}
A Pesquisa Nacional por Amostra de Domicílios (PNAD), realizada pelo Instituto Brasileiro de Geografia e Estatística (IBGE), é um conjunto de informações detalhadas sobre o cenário socioeconômico da sociedade brasileira. A partir dela, é possível extrair dados e informações das características gerais da população. A metodologia deste artigo consiste no processamento desses dados e a utilização de técnicas de aprendizado de máquina que possibilitem a previsão da probabilidade da evasão do aluno dadas as suas características de cunho social e econômico. Para tal, foram utilizadas as linguagens R, Python e a biblioteca Scikit-learn.

A base de dados foi obtida diretamente a partir do website do IBGE na seção PNAD Contínua e lida preliminarmente em R para interpretação dos dados de largura fixa a largura variável com o auxílio da biblioteca PNADcIBGE. Foram filtradas apenas os resultados pertinentes, a partir do de ano de 2016 até 2024, à analise de ocorrência de pessoas que são de São Paulo, não frequentam mais a escola, frequentaram escola alguma vez, cursaram como grau mais elevado o ensino médio regular ou 2º grau, e com idade até 18 anos. Com base nesses dados é possível rotular ocorrência de evasão ou conclusão do ensino médio como variável binária exclusiva dependente baseada no conclusão do curso.

Após o procedimento de filtragem, foi feita uma seleção das características, ou colunas, mais relevantes para explicar o fenômeno da evasão escolar. Características como sexo, cor ou raça, se já trabalhou ou estagiou por pelo menos 1 hora em alguma atividade remunerada em dinheiro, remunerada em mercadorias e bens, não remunerada ou atividade ocasional ("bico"), número de componentes do domicílio (exclusive as pessoas cuja condição no domicílio era pensionista, empregado doméstico ou parente do empregado doméstico), se recebe bolsa família ou outro auxílio governamental e rendimento domiciliar per capita
(habitual de todos os trabalhos e efetivo de outras fontes), as quais são algumas das características que mais se correlacionam qualitativamente ao abandono e à evasão escolar no ensino médio (FERREIRA; OLIVEIRA, 2020).

Foi gerado um arquivo CSV com os dados já filtrados e características selecionadas, e posteriormente este foi lido em Python para limpeza de dados faltantes, balanceamento de categorias pela técnica de sobreamostragem, e verificação dos tipos de variáveis com o objetivo de utilizá-lo como base de treinamento aos modelos de aprendizado de máquina. Foi também criada a coluna "evasao" codificada de forma binária para classificar o aluno como evasão (1) e não-evasão (0).


\subsection{Pré-processamento dos Dados}
Para simplificação e integridade dos modelos de regressão (OKEWOLE, 2012), as variáveis categóricas foram transformadas em variáveis binárias exclusivas. Variáveis numéricas foram padronizadas de acordo com a média das amostras e seu desvio padrão com base na fórmula:

\vspace{0.5cm}
\begin{equation}
\large Z = \frac{x-\mu}{s}
\end{equation}
\vspace{0.5cm}

Em que \textit{Z} é a novo valor da variável, \textit{x} é o valor atual da variável, $\mu$ representa a média da amostra para o grupo da variável (ou coluna) e \textit{s} é equivalente ao desvio-padrão do grupo da variável.

Para o treino e teste efetivo dos modelos, 85\% da base de dados foi destinada ao treinamento e o restante, 15\%, a testes para verificação das métricas. 

\subsection{Técnicas de Aprendizado de Máquina}
Este Trabalho apresenta três técnicas de aprendizado de máquina para efeito de modelagem probabilística, sendo elas Regressão Logística, Redes Neurais e Florestas Aleatórias de Classificação. Tais técnicas foram utilizadas devido a seus retornos baseados em estimativas em valores reais entre 0 e 1, que são consistentes com o modelo de probabilidades. Destaca-se que a escolha de diferentes técnicas reflete unicamente o objetivo de obter o mais acurado desempenho preditivo para o problema de previsão de probabilidades.

\subsection{Métricas de Desempenho}
Com o objetivo de extrair resultados acerca do desempenho dos modelos deste trabalho, foram utilizadas as fórmulas de precisão, revogação, acurácia e o índice F1 calculados com base na matriz de confusão proveniente do conjunto de testes com os modelos já treinados e selecionados os melhores via busca em grade de hiperparâmetros com validação cruzada em k-partições.


\subsubsection{Matriz de Confusão}
A matriz de confusão, assim como ressalta Franceschi (2019), é uma maneira de apresentar os valores reais e preditos pelo modelo com o intuito de mensurar os erros e acertos em cada classe de um determinado problema para se ter percepção dos seus desempenhos individuais.

Para este trabalho, dado que o problema segue a forma de classes binárias (evasão ou não-evasão), a matriz de confusão, seguindo o modelo de Franceschi (2019) é dada por:

\vspace{0.5cm}
\begin{equation}
\large
M = 
\begin{pmatrix}
VP & FP \\
FN & VN
\end{pmatrix}
\end{equation}
\vspace{0.5cm}

Conforme Castro e Braga (2011, apud Franceschi, 2019), a matriz principal da matriz representa os valores corretos preditos pelo modelo, isto é, \textit{VP} (Verdadeiro Positivo) e \textit{VN} (Verdadeiro Negativo). Os demais elementos expressam os erros retornados pelo modelo: \textit{FP} (Falso Positivo) e \textit{FN} (Falso Negativo).

Na representação deste trabalho, as linhas expressam as classes reais do problema, enquanto as colunas representam as classes preditas.

\subsubsection{Métricas Derivadas da Matriz de Confusão}
Dada a interpretação da matriz de confusão, pode-se, por meio de seus elementos, derivar métricas pertinentes à explicação do desempenho das classes e, por consequência, dos modelos deste trabalho, como demonstrado por Klén et al. (2024):

\vspace{0.5cm}
A taxa de precisão:
\begin{equation}
    \large \frac{VP}{VP+FP}
\end{equation}
\vspace{0.5cm}

\vspace{0.5cm}
A taxa de revocação:
\begin{equation}
    \large \frac{VP}{VP+FN}
\end{equation}
\vspace{0.5cm}

\vspace{0.5cm}
A taxa de acurácia:
\begin{equation}
    \large \frac{VP+VN}{VP+FP+VN+FN}
\end{equation}
\vspace{0.5cm}

\vspace{0.5cm}
E o índide F1:
\begin{equation}
    \large 2 \frac{P \cdot R}{P+R}
\end{equation}
\vspace{0.5cm}

Em que \textit{P} é a taxa de \textit{precisão} e \textit{R} a taxa de \textit{revocação}.

Acerca das métricas derivadas: a precisão representa os resultados classificados como verdadeiros positivos dentre os falsos e verdadeiros positivos, e mensura o quanto o algoritmo se distancia de falsos positivos; a taxa de revocação expressa a taxa de verdadeiros positivos em relação a verdadeiros positivos e falsos negativos, e demonstra o quanto o algoritmo separa corretamente verdadeiros positivos de falsos negativos; a acurácia representa a taxa de acertos bruta de todas as classes em todo o conjunto de teste; o índice F1, como descrito por Klén et al. (2024), é a média harmônica entre a precisão e a revocação. Para classes "negativas" (verdadeiro negativo), as mesmas lógicas da precisão e da revocação são implementadas com a ressalva de que são feitas em relação a falsos negativos e a falsos positivos respectivamente. 

\section{Resultados Parciais}

Com base nos dados coletados a partir da PNAD, as variáveis socioeconômicas, selecionadas neste presente trabalho, parecem não ser suficientes para determinar a evasão do aluno. O que se contrapõe aos trabalhos de Silva (2013) e Ferreira \& Oliveira (2020), que ressaltam tais variáveis como um dos principais motivos que levam ao abandono contínuo do ensino médio, em especial aos condições de trabalho de jovens em idade escolar.

Tal fato levanta a hipótese de que fatores socioeconômicos devem estar atrelados a fatores acadêmicos, como a presença ou notas, e fatores psicológicos, como a falta de interesse na escola, para uma real percepção e modelagem preditiva da evasão escolar. Por outro lado, também há a hipótese de que os fatores socioeconômicos que se correlacionam com a evasão não foram levados em consideração para a filtragem e modelagem deste trabalho.

A Pesquisa Nacional por Amostra de Domicílios (PNAD) apresenta um vasta base de dados acerca de aspectos socioeconômicos da população brasileira, contudo carece de determinantes acadêmicos específicos e questões que revelem a consideração do entrevistado acerca de sua vida no contexto acadêmico. 

Ao longo deste trabalho, as seguintes variáveis foram filtradas e tratadas para modelagem da evasão e treinamento posterior com os algoritmos de aprendizado de máquina:

\begin{table}
\caption{Descrição de Variáveis da Base de Dados}
\centering
\label{tab:variables}
\begin{tabular}{lp{8cm}}
\hline
Variável & Descrição \\ \hline
UF & Unidade da Federação do estudante \\
sexo & Sexo do estudante \\
raça & Raça/cor do estudante \\
idade & Idade em anos \\
frequenta\_escola & Indica se atualmente frequenta escola \\
frequentou\_escola & Indica se já frequentou escola em algum momento \\
curso\_frequentado & Tipo de curso frequentado (fundamental, médio, etc.) \\
terminou\_curso & Indica se concluiu o curso frequentado \\
V4001--V4004 & Variáveis da PNAD relacionadas ao trabalho e ocupação \\
condicao\_domicilio & Condição de ocupação do domicílio (próprio, alugado, cedido, etc.) \\
num\_pessoas\_domicilio & Número de pessoas no domicílio \\
bolsa\_familia & Recebimento de Bolsa Família \\
aux\_governo & Recebimento de outros auxílios governamentais \\
renda\_per\_capita & Renda domiciliar per capita \\
situacao\_domicilio & Situação do domicílio (urbano/rural) \\
tipo\_area\_dom & Tipo de área do domicílio \\
tipo\_domicilio & Tipo de construção do domicílio \\
tipo\_abastecimento\_de\_agua & Forma de abastecimento de água do domicílio \\
horas\_trabalhadas\_emp\_princ\_sem & Horas semanais trabalhadas no emprego principal \\
horas\_trabalhadas\_emp\_sec\_sem & Horas semanais trabalhadas no emprego secundário \\
horas\_trabalhadas\_outros\_emp\_sem & Horas semanais trabalhadas em outros empregos \\
procurou\_emprego\_ult\_30\_dias & Indica se procurou emprego nos últimos 30 dias \\
deseja\_trabalhar & Indica se deseja trabalhar atualmente \\
tempo\_procura\_emprego\_at1a & Tempo de procura por emprego: até 1 ano \\
tempo\_procura\_emprego\_1a2a & Tempo de procura por emprego: entre 1 e 2 anos \\
tempo\_procura\_emprego\_m2a & Tempo de procura por emprego: mais de 2 anos \\
num\_serie & Série escolar em que o estudante está matriculado \\
evasao & Indicador de evasão escolar (variável-alvo) \\ \hline
\end{tabular}
\end{table}


\subsubsection{Resultado do Desempenho dos Modelos}
A hipótese de que variáveis socioeconômicas não determinam de forma exclusiva o fenômeno da evasão no ensino médio pode ser constatada a partir do desempenho de cada classe com base nas métricas derivadas:

\vspace{0.5cm}
\begin{table}[htbp]
\centering
\label{tab:metricas}
\begin{tabular}{lcccc}
\hline
Classe        & Precisão & Revocação & Índice-F1 & Suporte \\ \hline
Não-evasão           & 0,86     & 0,77      & 0,81     & 2802    \\
Evasão           & 0,35     & 0,49      & 0,41     & 708     \\ \hline
Acurácia      &          &           & 0,72     & 3510    \\
Média Macro   & 0,60     & 0,63      & 0,61     & 3510    \\
Média Ponderada & 0,75   & 0,72      & 0,73     & 3510    \\ \hline
\end{tabular}
\caption{Métricas de desempenho do classificador Redes Neurais}
\end{table}
\vspace{0.5cm}

Com base na tabela, é possível perceber um alto desempenho em relação a Não-evasão, com uma precisão de 0,86, revocação de 0,77 e índice-f1 de 0,81. Contudo a classe que representa efetivamente a evasão tem precisão de 0,35 apenas, isto é, em todos os casos da classe Evasão (verdadeiros ou falsos), o algoritmo de Redes Neurais teve êxito em predizer corretamente apenas 35\% deles, e com falsos positivos (revocação) teve êxito na previsão de apenas 49\% dos casos. 

O algoritmo de Regressão Logística, apresentou desempenho semelhante para as classes:

\vspace{0.5cm}
\begin{table}[htbp]
\centering
\label{tab:metricas}
\begin{tabular}{lcccc}
\hline
Classe        & Precisão & Revocação & F1-Score & Suporte \\ \hline
Não-evasão           & 0,87     & 0,73      & 0,79     & 2802    \\
Evasão           & 0,35     & 0,57      & 0,43     & 708     \\ \hline
Acurácia      &          &           & 0,70     & 3510    \\
Média Macro   & 0,61     & 0,65      & 0,61     & 3510    \\
Média Ponderada & 0,76   & 0,70      & 0,72     & 3510    \\ \hline
\end{tabular}
\caption{Métricas de desempenho do classificador Regressão Logística}
\end{table}
\vspace{0.5cm}

O algoritmo de Floresta Aleatória se mostrou semelhante em relação aos outros modelos, com a ressalva de que apresentou desempenho ligeiramente pior em ambas as classes.

\vspace{0.5cm}
\begin{table}[htbp]
\centering
\label{tab:metricas2}
\begin{tabular}{lcccc}
\hline
Classe        & Precisão & Revocação & F1-Score & Suporte \\ \hline
Não-evasão           & 0,84     & 0,65      & 0,73     & 2802    \\
Evasão           & 0,27     & 0,53      & 0,36     & 708     \\ \hline
Acurácia      &          &           & 0,62     & 3510    \\
Média Macro   & 0,56     & 0,59      & 0,55     & 3510    \\
Média Ponderada & 0,73   & 0,62      & 0,66     & 3510    \\ \hline
\end{tabular}
\caption{Métricas de desempenho do classificador Floresta Aleatória}
\end{table}
\vspace{0.5cm}

Baseado nas tabelas de métricas dos diferentes modelos, apesar de evidenciado um baixo desempenho na classe de evasão, o algoritmo de Redes Neurais apresentou a maior acurácia global, contudo o modelo de Floresta Aleatória se mostrou o melhor na predição da classe evasão, com uma revocação e um índice-F1 superiores aos provenientes do modelo de Redes Neurais.

A natureza desbalanceada da base de dados bruta também pode ser um fator que interfere na capacidade do modelo de generalizar para ambas as classes, uma vez que para a classe de Evasão foi utilizado o método de sobreamostragem com a criação de registros sintéticos a partir de semelhança com k-vizinhos ($k=2$ neste trabalho)

\vspace{0.5cm}
\begin{table}[htbp]
\centering
\label{tab:proporcao}
\begin{tabular}{lc}
\hline
Classe & Proporção \\ \hline
Não-evasão  & 0,798 \\ 
Evasão      & 0,202 \\ \hline
\end{tabular}
\caption{Proporção das classes na base de dados}
\end{table}
\vspace{0.5cm}

Deste modo, pelo fato de aprender a partir de dados sintéticos, os diferentes modelos podem ter dificuldade de generalizar com dados reais, o que é verificado nos testes.


\pagebreak
\section{Cronograma}
O cronograma do projeto foi organizado de forma a contemplar as fases de pesquisa e os marcos de entrega definidos pela disciplina.
\begin{table}[h]
\centering
\begin{tabular}{p{3cm} p{7cm} p{5cm}}
\hline
\textbf{Período} & \textbf{Atividades principais} & \textbf{Marcos oficiais} \\
\hline
Agosto (até 20) & Revisão bibliográfica, definição do tema, objetivos e trabalhos relacionados & Entrega do plano de pesquisa (20/08) \\
Setembro        & Coleta de dados e pré-processamento & -- \\
Outubro (até 08) & Implementação inicial e testes dos modelos & Relatório parcial + vídeo (06--08/10) \\
Outubro (após 08) & Ajustes dos modelos, análises comparativas & -- \\
Novembro (até 19) & Consolidação dos resultados e redação final & Relatório final + apresentação parcial (17--19/11) \\
Novembro (após 19) & Revisão final e preparação da apresentação & Apresentação oficial (24--26/11) \\
\hline
\end{tabular}
\caption{Cronograma híbrido das atividades e entregas do projeto.}
\label{tab:cronograma}
\end{table}


%====================================================================


%See the guidelines for metadata and references:
%https://sol.sbc.org.br/journals/index.php/rbie/libraryFiles/downloadPublic/71
%====================================================================

\pagebreak
\nocite{*}
\printbibliography


\end{document}
